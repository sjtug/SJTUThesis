% 设置 biblatex 额外选项
% \PassOptionsToPackage{gbpub=false, gbtype=false}{biblatex}

% 载入 SJTUThesis 模版
% \documentclass[degree=doctor, zihao=-4, language=english, review]{src/sjtuthesis}
\documentclass[degree=master, zihao=-4]{src/sjtuthesis}
% \documentclass[degree=bachelor, openany, oneside]{src/sjtuthesis}
% \documentclass[degree=course, language=english, openright, twoside]{src/sjtuthesis}
% 选项
%   degree=[doctor|master|bachelor|course],     % 必选,学位类型
%   language=[chinese|english],                 % 可选(默认:chinese),论文的主要语言
%   bibstyle=[gb7714-2015|gb7714-2015ay|ieee],  % 可选(默认:gb7714-2015),参考文献样式
%   review,                                     % 可选(默认:关闭),盲审模式

% 所有其它可能用到的包都统一放到这里了,可以根据自己的实际添加或者删除。
\usepackage{src/sjtuthesis}

% 定义图片文件目录与扩展名
\graphicspath{{figure/}}
\DeclareGraphicsExtensions{.pdf,.eps,.png,.jpg,.jpeg}

% 导入参考文献数据库
\addbibresource{bib/thesis.bib}

% 信息录入,必须在导言区进行!
% !TEX root = ../thesis.tex

%TC:ignore

\title{上海交通大学学位论文 \LaTeX{} 模板示例文档}
\author{某\quad{}某}
\studentid{0010900990}
\supervisor{某某教授}
% \assisupervisor{某某教授}
\degree{工学硕士}
\major{某某专业}
\department{某某系}
\coursename{某某课程}
\date{2014年12月17日}
% \fund{国家 973 项目 (No. 2025CB000000) \\ 国家自然科学基金 (No. 81120250000)}
\keywords{上海交大, 饮水思源, 爱国荣校}

\entitle{A Sample Document for \LaTeX-based SJTU Thesis Template}
\enauthor{Mo Mo}
\ensupervisor{Prof. Mou Mou}
% \enassisupervisor{Prof. Uom Uom}
\endegree{Master of Engineering}
\enmajor{A Very Important Major}
\endepartment{Depart of XXX}
\endate{Dec. 17th, 2014}
% \enfund{National Basic Research Program of China (Grant No. 2025CB000000) \\
%   National Natural Science Foundation of China (Grant No. 81120250000)}
\enkeywords{SJTU, master thesis, XeTeX/LaTeX template}

%TC:endignore


% 自定义项目标签名称
% \sjtuSetLabel{
%   listfigure = {图\quad 录},
%   listtable  = {表\quad 录}
% }

\begin{document}

% 无编号内容:中英文论文封面、授权页
\maketitle
\makeDeclareOriginality[pdf/originality.pdf]
\makeDeclareAuthorization

% 使用罗马数字对前言编号
\frontmatter

% 摘要
\begin{abstract}

  上海交通大学是我国历史最悠久的高等学府之一,是教育部直属、教育部与上海市共建的全国重点大学,是国家 “七五”、“八五”重点建设和“211工程”、“985工程”的首批建设高校。经过113年的不懈努力,上海交通大学已经成为一所“综合性、研究型、国际化”的国内一流、国际知名大学,并正在向世界一流大学稳步迈进。
  学校现有本科专业67个,涵盖经济学、法学、文学、理学、工学、农学、医学和管理学等8个学科门类;拥有工科物理、工科数学和电工电子等3个国家工科基础课程教学基地,生命科学和集成电路等2个国家人才培养基地和教育部大学生文化素质教育基地,以及国家生物学理科人才培养基地;有国家级实验教学示范中心5个,上海市实验教学示范中心4个;有国家级教学团队5个,上海市级教学团队9各;有国家级教学名师奖获得者6人,上海市教学名师奖获得者32人;有国家级精品课程40门,上海市精品课程100门;有国家级双语示范课程5门;2005年和2009年,作为第一完成单位,共获得国家级教学成果22项、上海市级教学成果105项。

  \keywords{\large 上海交大 \quad 饮水思源 \quad 爱国荣校}
\end{abstract}

\begin{englishabstract}

  Shanghai Jiao Tong University (SJTU), directly subordinate to the Ministry of Education, is a key university in China, jointly run by the Ministry and Shanghai Municipality.
  SJTU has beautiful campuses, occupying an area of more than 200 hectare in total, and possesses plenty of advanced teaching and research equipment and facilities. Now, it has six campuses, the Xuhui, the Minhang, the Qibao, the Shangzhong Road , the Fahuazheng Road and the Chongqing Road(south). Over the past decade, the number of students in SJTU has grown from 5,000 to more than 38,000, the floorage of various buildings from 230,000 square meters to 800,000 square meters, and the area of campuses from 40ha to 200ha. Apart from the major buildings such as the Lecture Buildings, Laboratory Buildings, Dormitories and Gymnasiums, SJTU also has the Bao Zhaolong Library which is well-known throughout the country. Various laboratories, including university central laboratories such as "Computer Center" and "Audio-visual Education Center" are equipped with advanced research and teaching equipment and facilities.

  \englishkeywords{\large SJTU, master thesis, LaTeX template}
\end{englishabstract}


% 目录、插图目录、表格目录
\tableofcontents
\listoffigures
\listoftables
\listofalgorithms

% 主要符号、缩略词对照表
% !TEX root = ../thesis.tex

%TC:ignore

\begin{nomenclature}{rl}
\label{chap:symb}
  $\epsilon$     & 介电常数 \\  
  $\mu$ 		& 磁导率 \\
  $\epsilon$     & 介电常数 \\
  $\mu$ 		& 磁导率 \\
  $\epsilon$     & 介电常数 \\
  $\mu$ 		& 磁导率 \\
  $\epsilon$ 	& 介电常数 \\
  $\mu$ 		& 磁导率 \\
  $\epsilon$     & 介电常数 \\
  $\mu$ 		& 磁导率 \\
  $\epsilon$     & 介电常数 \\
  $\mu$ 		& 磁导率 \\
  $\epsilon$     & 介电常数 \\
  $\mu$ 		& 磁导率 \\
  $\epsilon$ 	& 介电常数 \\
  $\mu$ 		& 磁导率 \\
  $\epsilon$     & 介电常数 \\
  $\mu$ 		& 磁导率 \\
  $\epsilon$     & 介电常数 \\
  $\mu$ 		& 磁导率 \\
  $\epsilon$     & 介电常数 \\
  $\mu$ 		& 磁导率 \\
  $\epsilon$ 	& 介电常数 \\
  $\mu$ 		& 磁导率 \\
  $\epsilon$     & 介电常数 \\
  $\mu$ 		& 磁导率 \\
  $\epsilon$     & 介电常数 \\
  $\mu$ 		& 磁导率 \\
  $\epsilon$     & 介电常数 \\
  $\mu$ 		& 磁导率 \\
  $\epsilon$ 	& 介电常数 \\
  $\mu$ 		& 磁导率 \\
  $\epsilon$     & 介电常数 \\
  $\mu$ 		& 磁导率 \\
  $\epsilon$     & 介电常数 \\
  $\mu$ 		& 磁导率 \\
  $\epsilon$     & 介电常数 \\
  $\mu$ 		& 磁导率 \\
  $\epsilon$ 	& 介电常数 \\
  $\mu$ 		& 磁导率 \\
  $\epsilon$     & 介电常数 \\
  $\mu$ 		& 磁导率 \\
  $\epsilon$     & 介电常数 \\
  $\mu$ 		& 磁导率 \\
  $\epsilon$     & 介电常数 \\
  $\mu$ 		& 磁导率 \\
  $\epsilon$ 	& 介电常数 \\
  $\mu$ 		& 磁导率 \\
  $\epsilon$     & 介电常数 \\
  $\mu$ 		& 磁导率 \\
  $\epsilon$     & 介电常数 \\
  $\mu$ 		& 磁导率 \\
  $\epsilon$     & 介电常数 \\
  $\mu$ 		& 磁导率 \\
\end{nomenclature}

%TC:endignore


% 使用阿拉伯数字对正文编号
\mainmatter

% 正文内容
%# -*- coding: utf-8-unix -*-
%%==================================================
%% chapter01.tex for SJTU Master Thesis
%%==================================================

%\bibliographystyle{sjtu2}%[此处用于每章都生产参考文献]
\chapter{这是什么}
\label{chap:intro}

这是上海交通大学(非官方)学位论文 \LaTeX 模板,当前版本是 \version 。

最早的一版学位模板是一位热心的物理系同学制作的。
那份模板参考了自动化所学位论文模板,使用了CASthesis.cls文档类,中文字符处理则采用当时最为流行的 \CJKLaTeX 方案。
我根据交大研究生院对学位论文的要求
\footnote{\url{http://www.gs.sjtu.edu.cn/policy/fileShow.ahtml?id=130}}
,结合少量个人审美喜好,完成了一份基本可用的交大 \LaTeX 学位论文模板。
但是,搭建一个 \CJKLaTeX 环境并不简单,单单在Linux下配置环境和添加中文字体,就足够让新手打退堂鼓。
在William Wang的建议下,我开始着手把模板向 \XeTeX 引擎移植。
他完成了最初的移植,多亏了他出色的工作,后续的改善工作也得以顺利进行。

随着我对 \LaTeX 系统认知增加,我又断断续续做了一些完善模板的工作,在原有硕士学位论文模板的基础上完成了交大学士和博士学位论文模板。

现在,交大学位论文模板SJTUTHesis代码在github
\footnote{\url{https://github.com/weijianwen/SJTUThesis}}
上维护。
你可以\href{https://github.com/weijianwen/SJTUThesis/issues}{在github上开issue}
、或者在\href{https://bbs.sjtu.edu.cn/bbsdoc?board=TeX_LaTeX}{水源LaTeX版}发帖来反映遇到的问题。

\section{使用模板}

\subsection{准备工作}
\label{sec:requirements}

要使用这个模板撰写学位论文,需要在\emph{TeX系统}、\emph{中英文字体}、\emph{TeX技能}上有所准备。

\begin{itemize}[noitemsep,topsep=0pt,parsep=0pt,partopsep=0pt]
	\item {\TeX}系统:所使用的{\TeX}系统要支持 \XeTeX 引擎,且带有ctex 2.x宏包,以2015年的\emph{完整}TeXLive、MacTeX发行版为佳。
	\item 中英文字体:操作系统中需要安装\footnote{在Windows、Mac OS X 以及 Linux 上安装额外的字体,可以参考\href{https://www.searchfreefonts.com/articles/how-to-install-fonts.htm}{“How to install fonts?”}。
}TeX Gyre Termes字体\footnote{\url{http://www.gust.org.pl/projects/e-foundry/tex-gyre/termes}}和四款Adobe中文字体
\footnote{请从合法渠道获得Adobe字体。}:AdobeSongStd、AdobeKaitiStd、AdobeHeitiStd、AdobeFangsongStd。
	\item TeX技能:尽管提供了对模板的必要说明,但这不是一份“ \LaTeX 入门文档”。在使用前请先通读其他入门文档。
	\item 针对Windows用户的额外需求:学位论文模本分别使用git和GNUMake进行版本控制和构建,建议从Cygwin\footnote{\url{http://cygwin.com}}安装这两个工具。
\end{itemize}

\subsection{模板选项}
\label{sec:thesisoption}

sjtuthesis提供了一些常用选项,在thesis.tex在导入sjtuthesis模板类时,可以组合使用。
这些选项包括:

\begin{itemize}[noitemsep,topsep=0pt,parsep=0pt,partopsep=0pt]
\item 学位类型:bachelor(学位)、master(硕士)、doctor(博士),是必选项。
\item 中国字体:adobefonts(Adobe中文字体)、winfonts(使用Windows下的中文字体,该选项未在Linux/Mac下测试)。
\item 正文字号:cs4size(小四)、c5size(五号)。
\item 盲审选项:使用review选项后,论文作者、学号、导师姓名、致谢、发表论文和参与项目将被隐去。
\end{itemize}

\subsection{编译模板}
\label{sec:process}

模板默认使用GNUMake构建,GNUMake将调用latemk工具自动完成模板多轮编译:

\begin{lstlisting}[basicstyle=\small\ttfamily, caption={编译模板}, numbers=none]
make clean thesis.pdf
\end{lstlisting}

若需要生成包含“原创性声明扫描件”的学位论文文档,请将扫描件保存为statement.pdf,然后调用make生成submit.pdf。

\begin{lstlisting}[basicstyle=\small\ttfamily, caption={生成用于提交的学位论文}, numbers=none]
make clean submit.pdf
\end{lstlisting}

编译失败时,可以尝试手动逐次编译,定位故障。

\begin{lstlisting}[basicstyle=\small\ttfamily, caption={手动逐次编译}, numbers=none]
xelatex -no-pdf thesis
biber --debug thesis
xelatex thesis
xelatex thesis
\end{lstlisting}

\subsection{模板文件布局}
\label{sec:layout}

\begin{lstlisting}[basicstyle=\small\ttfamily,caption={模板文件布局},label=layout,float,numbers=none]
├── LICENSE
├── Makefile
├── README.md
├── bib
│   ├── chap1.bib
│   └── chap2.bib
├── bst
│   └── GBT7714-2005NLang.bst
├── figure
│   ├── chap2
│   │   ├── sjtulogo.eps
│   │   ├── sjtulogo.jpg
│   │   ├── sjtulogo.pdf
│   │   └── sjtulogo.png
│   └── sjtubanner.png
├── sjtuthesis.cfg
├── sjtuthesis.cls
├── statement.pdf
├── submit.pdf
├── tex
│   ├── abstract.tex
│   ├── ack.tex
│   ├── app_cjk.tex
│   ├── app_eq.tex
│   ├── app_log.tex
│   ├── chapter01.tex
│   ├── chapter02.tex
│   ├── chapter03.tex
│   ├── conclusion.tex
│   ├── id.tex
│   ├── patents.tex
│   ├── projects.tex
│   ├── pub.tex
│   └── symbol.tex
└── thesis.tex
\end{lstlisting}

本节介绍学位论文模板中木要文件和目录的功能。

\subsubsection{格式控制文件}
\label{sec:format}

格式控制文件控制着论文的表现形式,包括以下几个文件:
sjtuthesis.cfg, sjtuthesis.cls和GBT7714-2005NLang.bst。
其中,“cfg”和“cls”控制论文主体格式,“bst”控制参考文献条目的格式,

\subsubsection{主控文件thesis.tex}
\label{sec:thesistex}

主控文件thesis.tex的作用就是将你分散在多个文件中的内容“整合”成一篇完整的论文。
使用这个模板撰写学位论文时,你的学位论文内容和素材会被“拆散”到各个文件中:
譬如各章正文、各个附录、各章参考文献等等。
在thesis.tex中通过“include”命令将论文的各个部分包含进来,从而形成一篇结构完成的论文。
对模板定制时引入的宏包,建议放在导言区。

\subsubsection{各章源文件tex}
\label{sec:thesisbody}

这一部分是论文的主体,是以“章”为单位划分的,包括:

\begin{itemize}[noitemsep,topsep=0pt,parsep=0pt,partopsep=0pt]
	\item 中英文摘要(abstract.tex)。前言(frontmatter)的其他部分,中英文封面、原创性声明、授权信息在sjtuthesis.cls中定义,不单独分离为tex文件。
不单独弄成文件。
	\item 正文(mainmatter)——学位论文正文的各章内容,源文件是chapter\emph{xxx}.tex。
	\item 附录(app\emph{xx}.tex)、致谢(thuanks.tex)、攻读学位论文期间发表的学术论文目录(pub.tex)、个人简历(resume.tex)组成正文后的部分(backmatter)。
参考文献列表由bibtex插入,不作为一个单独的文件。
\end{itemize}

\subsubsection{图片文件夹figure}
\label{sec:fig}

figure文件夹放置了需要插入文档中的图片文件(支持PNG/JPG/PDF/EPS格式的图片),可以在按照章节划分子目录。
模板文件中使用\verb|\graphicspath|命令定义了图片存储的顶层目录,在插入图片时,顶层目录名“figure”可省略。

\subsubsection{参考文献数据库bib}
\label{sec:bib}

目前参考文件数据库目录只存放一个参考文件数据库thesis.bib。
关于参考文献引用,可参考第\ref{chap:example}章中的例子。


% !TeX root = ../main.tex

\chapter{浮动体}

\section{插图}

插图功能是利用 \TeX{} 的特定编译程序提供的机制实现的,不同的编译程序支持不同的图
形方式。有的同学可能听说“\LaTeX{} 只支持 EPS”,事实上这种说法是不准确的。\XeTeX{}
可以很方便地插入 EPS、PDF、PNG、JPEG 格式的图片。

一般图形都是处在浮动环境中。之所以称为浮动是指最终排版效果图形的位置不一定与源文
件中的位置对应,这也是刚使用 \LaTeX{} 同学可能遇到的问题。如果要强制固定浮动图形
的位置,请使用 \pkg{float} 宏包,它提供了 \texttt{[H]} 参数。

\subsection{单个图形}

图要有图题,研究生图题采用中英文对照,并置于图的编号之后,图的编号和图题应置于图
下方的居中位置。引用图应在图题右上角标出文献来源。文中必须有关于本插图的提示,如
“见图~\ref{fig:energy-distrib}”、“如图~\ref{fig:energy-distrib} 所示”等。该页空
白不够排写该图整体时,则可将其后文字部分提前排写,将图移到次页。

\begin{figure}[!htp]
  \centering
  \begin{tikzpicture}
    \begin{axis}[
      width=12cm,
      height=9cm,
      xmin=0, xmax=7,
      xlabel={$r$ (\unit{\milli\metre})},
      ymin=-1000, ymax=11000,
      ylabel={Energy (\unit[per-mode=symbol]{\watt\per\cubic\metre})},
      scaled ticks=false,
      tick label style={
        /pgf/number format/1000 sep=,
        font={\zihao{-5}},
      },
      minor tick num=1,
      tick pos=left,
      tick align=outside,
      tick style={thin,black},
    ]
      \addplot [only marks,mark=square*] 
        table [x={radial}, y={energy}, col sep=comma] 
        {./assets/energy-distrib.csv};
      \node at (2,6000) 
        {$q_{v}=\dfrac{\sigma\omega^{2}|\mathbf{A}|^{2}}{2}$};
    \end{axis}
  \end{tikzpicture}
  \bicaption{内热源沿径向的分布}{Energy distribution along radial}
  \label{fig:energy-distrib}
\end{figure}

\subsection{多个图形}

简单插入多个图形的例子如图~\ref{fig:SRR} 所示。这两个水平并列放置的子图共用一个
图形计数器,没有各自的子图题。

\begin{figure}[!htp]
  \centering
  \includegraphics[height=2.5cm]{example-image-a.pdf}%
  \hfil\hfil%
  \includegraphics[height=2.5cm]{example-image-b.pdf}
  \bicaption{中文题图}{English caption}
  \label{fig:SRR}
\end{figure}

如果多个图形相互独立,并不共用一个图形计数器,那么用 \texttt{minipage} 或者
\texttt{parbox} 就可以,如图~\ref{fig:parallel1} 与图~\ref{fig:parallel2}。

\begin{figure}[!htp]
  \centering
  \begin{minipage}{0.5\textwidth}
    \centering
    \includegraphics[height=2.5cm]{example-image-a.pdf}
    \caption{并排第一个图}
    \label{fig:parallel1}
  \end{minipage}%
  \begin{minipage}{0.5\textwidth}
    \centering
    \includegraphics[height=2.5cm]{example-image-b.pdf}
    \caption{并排第二个图}
    \label{fig:parallel2}
  \end{minipage}
\end{figure}

如果要为共用一个计数器的多个子图添加子图题,建议使用较新的 \pkg{subcaption} 宏
包,不建议使用 \pkg{subfigure} 或 \pkg{subfig} 等宏包。

推荐使用 \pkg{subcaption} 宏包的 \cs{subcaptionbox} 并排子图,子图题置于子图之
下,子图号用 a)、b) 等表示。也可以使用 \pkg{subcaption} 宏包的 \cs{subcaption}
(放在 minipage中,用法同 \cs{caption})。

\pkg{subcaption} 宏包也提供了 \pkg{subfigure} 和 \pkg{subtable} 环境,如
图~\ref{fig:subfigure}。

\begin{figure}[!htp]
  \centering
  \begin{subfigure}{0.45\textwidth}
    \centering
    \includegraphics[height=2.5cm]{example-image-a.pdf}
    \caption{子图甲}
  \end{subfigure}%
  \hfil\hfil%
  \begin{subfigure}{0.45\textwidth}
    \centering
    \includegraphics[height=2cm]{example-image-b.pdf}
    \caption{子图乙。注意这个图略矮些,subfigure 中同一行的子图在顶端对齐。}
  \end{subfigure}
  \caption{包含子图题的范例(使用 subfigure)}
  \label{fig:subfigure}
\end{figure}

搭配 \pkg{bicaption} 宏包时,可以启用 \cs{subcaptionbox} 和 \cs{subcaption} 的双
语变种 \cs{bisubcaptionbox} 和 \cs{bisubcaption},如图~\ref{fig:bisubcaptionbox}
所示。

\begin{figure}[!hbtp]
  \centering
  \bisubcaptionbox{$R_3 = 1.5\text{mm}$ 时轴承的压力分布云图}%
                  {Pressure contour of bearing when $R_3 = 1.5\text{mm}$}%
                  [0.4\textwidth]{\includegraphics[height=2.5cm]{example-image-a.pdf}}%
  \hfil\hfil%
  \bisubcaptionbox{$R_3 = 2.5\text{mm}$ 时轴承的压力分布云图}%
                  {Pressure contour of bearing when $R_3 = 2.5\text{mm}$}%
                  [0.4\textwidth]{\includegraphics[height=2.5cm]{example-image-b.pdf}}
  \bicaption{包含子图题的范例(使用 subcaptionbox)}
            {Example with subcaptionbox}
  \label{fig:bisubcaptionbox}
\end{figure}


\section{表格}

\subsection{基本表格}

编排表格应简单明了,表达一致,明晰易懂,表文呼应、内容一致。表题置于表上,研究生
学位论文可以用中、英文两种文字居中排写,中文在上,也可以只用中文。

表格的编排建议采用国际通行的三线表\footnote{三线表,以其形式简洁、功能分明、阅读
方便而在科技论文中被推荐使用。三线表通常只有 3 条线,即顶线、底线和栏目线,没有
竖线。}。三线表可以使用 \pkg{booktabs} 提供的 \cs{toprule}、\cs{midrule} 和
\cs{bottomrule}。它们与 \pkg{longtable} 能很好的配合使用。

\begin{table}[!hpt]
  \caption[一个颇为标准的三线表]{一个颇为标准的三线表\footnotemark}
  \label{tab:firstone}
  \centering
  \begin{tabular}{@{}llr@{}} \toprule
    \multicolumn{2}{c}{Item} \\ \cmidrule(r){1-2}
    Animal & Description & Price (\$)\\ \midrule
    Gnat  & per gram  & 13.65 \\
          & each      & 0.01 \\
    Gnu   & stuffed   & 92.50 \\
    Emu   & stuffed   & 33.33 \\
    Armadillo & frozen & 8.99 \\ \bottomrule
  \end{tabular}
\end{table}
\footnotetext{这个例子来自
  \href{https://mirrors.sjtug.sjtu.edu.cn/ctan/macros/latex/contrib/booktabs/booktabs.pdf}%
  {《Publication quality tables in LaTeX》}(\pkg{booktabs} 宏包的文档)。这也是
  一个在表格中使用脚注的例子,请留意与 \pkg{threeparttable} 实现的效果有何不
  同。}

\subsection{复杂表格}

我们经常会在表格下方标注数据来源,或者对表格里面的条目进行解释。可以用
\pkg{threeparttable} 实现带有脚注的表格,如表~\ref{tab:footnote}。

\begin{table}[!htpb]
  \bicaption{一个带有脚注的表格的例子}{A Table with footnotes}
  \label{tab:footnote}
  \centering
  \begin{threeparttable}[b]
     \begin{tabular}{ccd{4}cccc}
      \toprule
      \multirow{2}*{total} & \multicolumn{2}{c}{20\tnote{a}} & \multicolumn{2}{c}{40} & \multicolumn{2}{c}{60} \\
      \cmidrule(lr){2-3}\cmidrule(lr){4-5}\cmidrule(lr){6-7}
      & www & \multicolumn{1}{c}{k} & www & k & www & k \\ % 使用说明符 d 的列会自动进入数学模式,使用 \multicolumn 对文字表头做特殊处理
      \midrule
      & $\underset{(2.12)}{4.22}$ & 120.0140\tnote{b} & 333.15 & 0.0411 & 444.99 & 0.1387 \\
      & 168.6123 & 10.86 & 255.37 & 0.0353 & 376.14 & 0.1058 \\
      & 6.761    & 0.007 & 235.37 & 0.0267 & 348.66 & 0.1010 \\
      \bottomrule
    \end{tabular}
    \begin{tablenotes}
    \item [a] the first note.
    \item [b] the second note.
    \end{tablenotes}
  \end{threeparttable}
\end{table}

如某个表需要转页接排,可以用 \pkg{longtable} 实现。接排时表题省略,表头应重复书
写,并在右上方写“续表 xx”,如表~\ref{tab:performance}。

\begin{ThreePartTable}
  \begin{TableNotes}
    \item[a] 一个脚注
    \item[b] 另一个脚注
  \end{TableNotes}
  \begin{longtable}[c]{c*{6}{r}}
    \bicaption{实验数据}{Experimental data}
    \label{tab:performance} \\
    \toprule
    测试程序 & \multicolumn{1}{c}{正常运行} & \multicolumn{1}{c}{同步}
      & \multicolumn{1}{c}{检查点} & \multicolumn{1}{c}{卷回恢复}
      & \multicolumn{1}{c}{进程迁移} & \multicolumn{1}{c}{检查点} \\
    & \multicolumn{1}{c}{时间 (s)} & \multicolumn{1}{c}{时间 (s)}
      & \multicolumn{1}{c}{时间 (s)} & \multicolumn{1}{c}{时间 (s)}
      & \multicolumn{1}{c}{时间 (s)} &  文件(KB)\\
    \midrule
    \endfirsthead
    \multicolumn{7}{l}{\textbf{续表~\thetable}} \\
    % 英语论文:\multicolumn{7}{r}{\textbf{Table~\thetable~(continued)}} \\
    \toprule
    测试程序 & \multicolumn{1}{c}{正常运行} & \multicolumn{1}{c}{同步}
      & \multicolumn{1}{c}{检查点} & \multicolumn{1}{c}{卷回恢复}
      & \multicolumn{1}{c}{进程迁移} & \multicolumn{1}{c}{检查点} \\
    & \multicolumn{1}{c}{时间 (s)} & \multicolumn{1}{c}{时间 (s)}
      & \multicolumn{1}{c}{时间 (s)} & \multicolumn{1}{c}{时间 (s)}
      & \multicolumn{1}{c}{时间 (s)}&  文件(KB)\\
    \midrule
    \endhead
    \hline
    \multicolumn{7}{r}{续下页}
    \endfoot
    \insertTableNotes
    \endlastfoot
    CG.A.2 & 23.05 & 0.002 & 0.116 & 0.035 & 0.589 & 32491 \\
    CG.A.4 & 15.06 & 0.003 & 0.067 & 0.021 & 0.351 & 18211 \\
    CG.A.8 & 13.38 & 0.004 & 0.072 & 0.023 & 0.210 & 9890 \\
    CG.B.2 & 867.45 & 0.002 & 0.864 & 0.232 & 3.256 & 228562 \\
    CG.B.4 & 501.61 & 0.003 & 0.438 & 0.136 & 2.075 & 123862 \\
    CG.B.8 & 384.65 & 0.004 & 0.457 & 0.108 & 1.235 & 63777 \\
    MG.A.2 & 112.27 & 0.002 & 0.846 & 0.237 & 3.930 & 236473 \\
    MG.A.4 & 59.84 & 0.003 & 0.442 & 0.128 & 2.070 & 123875 \\
    MG.A.8 & 31.38 & 0.003 & 0.476 & 0.114 & 1.041 & 60627 \\
    MG.B.2 & 526.28 & 0.002 & 0.821 & 0.238 & 4.176 & 236635 \\
    MG.B.4 & 280.11 & 0.003 & 0.432 & 0.130 & 1.706 & 123793 \\
    MG.B.8 & 148.29 & 0.003 & 0.442 & 0.116 & 0.893 & 60600 \\
    LU.A.2 & 2116.54 & 0.002 & 0.110 & 0.030 & 0.532 & 28754 \\
    LU.A.4 & 1102.50 & 0.002 & 0.069 & 0.017 & 0.255 & 14915 \\
    LU.A.8 & 574.47 & 0.003 & 0.067 & 0.016 & 0.192 & 8655 \\
    LU.B.2 & 9712.87 & 0.002 & 0.357 & 0.104 & 1.734 & 101975 \\
    LU.B.4 & 4757.80 & 0.003 & 0.190 & 0.056 & 0.808 & 53522 \\
    LU.B.8 & 2444.05 & 0.004 & 0.222 & 0.057 & 0.548 & 30134 \\
    EP.A.2 & 123.81 & 0.002 & 0.010 & 0.003 & 0.074 & 1834 \\
    EP.A.4 & 61.92 & 0.003 & 0.011 & 0.004 & 0.073 & 1743 \\
    EP.A.8 & 31.06 & 0.004 & 0.017 & 0.005 & 0.073 & 1661 \\
    EP.B.2 & 495.49 & 0.001 & 0.009 & 0.003 & 0.196 & 2011 \\
    EP.B.4 & 247.69 & 0.002 & 0.012 & 0.004 & 0.122 & 1663 \\
    EP.B.8 & 126.74 & 0.003 & 0.017 & 0.005 & 0.083 & 1656 \\
    SP.A.2 & 123.81 & 0.002 & 0.010 & 0.003 & 0.074 & 1854 \\
    SP.A.4 & 51.92 & 0.003 & 0.011 & 0.004 & 0.073 & 1543 \\
    SP.A.8 & 31.06 & 0.004 & 0.017 & 0.005 & 0.073 & 1671 \\
    SP.B.2 & 495.49 & 0.001 & 0.009 & 0.003 & 0.196 & 2411 \\
    SP.B.4 \tnote{a} & 247.69 & 0.002 & 0.014 & 0.006 & 0.152 & 2653 \\
    SP.B.8 \tnote{b} & 126.74 & 0.003 & 0.017 & 0.005 & 0.082 & 1755 \\
    \bottomrule
  \end{longtable}
\end{ThreePartTable}

\section{算法环境}

算法环境可以使用 \pkg{algorithms} 宏包或者较新的 \pkg{algorithm2e} 实现。
算法~\ref{algo:algorithm} 是一个使用 \pkg{algorithm2e} 的例子。关于排版算法环境
的具体方法,请阅读相关宏包的官方文档。

\begin{algorithm}[htb]
  \caption{算法示例}
  \label{algo:algorithm}
  \SetAlgoLined
  \KwData{this text}
  \KwResult{how to write algorithm with \LaTeXe }

  initialization\;
  \While{not at end of this document}{
    read current\;
    \eIf{understand}{
      go to next section\;
      current section becomes this one\;
    }{
      go back to the beginning of current section\;
    }
  }
\end{algorithm}

\section{代码环境}

我们可以在论文中插入算法,但是不建议插入大段的代码。如果确实需要插入代码,建议使
用 \pkg{listings} 宏包。

\begin{codeblock}[language=C]
#include <stdio.h>
#include <unistd.h>
#include <sys/types.h>
#include <sys/wait.h>

int main() {
  pid_t pid;

  switch ((pid = fork())) {
  case -1:
    printf("fork failed\n");
    break;
  case 0:
    /* child calls exec */
    execl("/bin/ls", "ls", "-l", (char*)0);
    printf("execl failed\n");
    break;
  default:
    /* parent uses wait to suspend execution until child finishes */
    wait((int*)0);
    printf("is completed\n");
    break;
  }

  return 0;
}
\end{codeblock}

% !TEX root = ../thesis.tex

\chapter{数学与引用文献的标注}

\section{数学}

\subsection{数字和单位}

宏包 \pkg{siunitx} 提供了更好的数字和单位支持:
\begin{itemize}
  \item \num{12345.67890}
  \item \num{1+-2i}
  \item \num{.3e45}
  \item \num{1.654 x 2.34 x 3.430}
  \item \si{kg.m.s^{-1}}
  \item \si{\micro\meter} $\si{\micro\meter}$
  \item \si{\ohm} $\si{\ohm}$
  \item \numlist{10;20}
  \item \numlist{10;20;30}
  \item \SIlist{0.13;0.67;0.80}{\milli\metre}
  \item \numrange{10}{20}
  \item \SIrange{10}{20}{\degreeCelsius}
\end{itemize}

\subsection{数学符号和公式}

微分符号 $\dif$ 应使用正体,本模板提供了 \cs{dif} 命令。除此之外,模板还提供了一
些命令方便使用:
\begin{itemize}
  \item 圆周率 $\uppi$:\verb|\uppi|
  \item 自然对数的底 $\upe$:\verb|\upe|
  \item 虚数单位 $\upi$, $\upj$:\verb|\upi| \verb|\upj|
\end{itemize}

公式应另起一行居中排版。公式后应注明编号,按章顺序编排,编号右端对齐。
\begin{equation}
  \upe^{\upi\uppi} + 1 = 0,
\end{equation}
\begin{equation}
  \frac{\dif^2 u}{\dif t^2} = \int f(x) \dif x.
\end{equation}

公式末尾是需要添加标点符号的,至于用逗号还是句号,取决于公式下面一句是接着公式说的,还是另起一句。
\begin{equation}
		\frac{2h}{\pi}\int_{0}^{\infty}\frac{\sin\left( \omega\delta \right)}{\omega}
		\cos\left( \omega x \right) \dif\omega = 
		\begin{cases}
				h, \ \left| x \right| < \delta, \\
				\frac{h}{2}, \ x = \pm \delta, \\
				0, \ \left| x \right| > \delta.
		\end{cases}
\end{equation}
公式较长时最好在等号“$=$”处转行。
\begin{align}
    & I (X_3; X_4) - I (X_3; X_4 \mid X_1) - I (X_3; X_4 \mid X_2) \nonumber \\
  = & [I (X_3; X_4) - I (X_3; X_4 \mid X_1)] - I (X_3; X_4 \mid \tilde{X}_2) \\
  = & I (X_1; X_3; X_4) - I (X_3; X_4 \mid \tilde{X}_2).
\end{align}

如果在等号处转行难以实现,也可在 $+$、$-$、$\times$、$\div$运算符号处转行,转行
时运算符号仅书写于转行式前,不重复书写。
\begin{multline}
  \frac{1}{2} \Delta (f_{ij} f^{ij}) =
    2 \left(\sum_{i<j} \chi_{ij}(\sigma_{i} - \sigma_{j})^{2}
    + f^{ij} \nabla_{j} \nabla_{i} (\Delta f) \right. \\
  \left. + \nabla_{k} f_{ij} \nabla^{k} f^{ij} +
    f^{ij} f^{k} \left[2\nabla_{i}R_{jk}
    - \nabla_{k} R_{ij} \right] \vphantom{\sum_{i<j}} \right).
\end{multline}

\subsection{定理环境}

示例文件中使用 \pkg{ntheorem} 宏包配置了定理、引理和证明等环境。用户也可以使用
\pkg{amsthm} 宏包。

这里举一个“定理”和“证明”的例子。
\begin{theorem}[留数定理]
\label{thm:res}
  假设 $U$ 是复平面上的一个单连通开子集,$a_1, \ldots, a_n$ 是复平面上有限个点,
  $f$ 是定义在 $U \backslash \{a_1, \ldots, a_n\}$ 上的全纯函数,如果 $\gamma$
  是一条把 $a_1, \ldots, a_n$ 包围起来的可求长曲线,但不经过任何一个 $a_k$,并且
  其起点与终点重合,那么:

  \begin{equation}
    \label{eq:res}
    \ointop_\gamma f(z)\, \dif z = 2\uppi \upi \sum_{k=1}^n \operatorname{I}(\gamma, a_k) \operatorname{Res}(f, a_k).
  \end{equation}

  如果 $\gamma$ 是若尔当曲线,那么 $\operatorname{I}(\gamma, a_k) = 1$,因此:

  \begin{equation}
    \label{eq:resthm}
    \ointop_\gamma f(z)\, \dif z = 2\uppi \upi \sum_{k=1}^n \operatorname{Res}(f, a_k).
  \end{equation}

  在这里,$\operatorname{Res}(f, a_k)$ 表示 $f$ 在点 $a_k$ 的留数,
  $\operatorname{I}(\gamma, a_k)$ 表示 $\gamma$ 关于点 $a_k$ 的卷绕数。卷绕数是
  一个整数,它描述了曲线 $\gamma$ 绕过点 $a_k$ 的次数。如果 $\gamma$ 依逆时针方
  向绕着 $a_k$ 移动,卷绕数就是一个正数,如果 $\gamma$ 根本不绕过 $a_k$,卷绕数
  就是零。

  定理~\ref{thm:res} 的证明。

  \begin{proof}
    首先,由……

    其次,……

    所以……
  \end{proof}
\end{theorem}

\section{引用文献的标注}

按照教务处的要求,参考文献外观应符合国标 GB/T 7714 的要求。模版使用 \BibLaTeX\
配合 \pkg{biblatex-gb7714-2015} 样式包
\footnote{\url{https://www.ctan.org/pkg/biblatex-gb7714-2015}}
控制参考文献的输出样式,后端采用 \pkg{biber} 管理文献。

请注意 \pkg{biblatex-gb7714-2015} 宏包 2016 年 9 月才加入 CTAN,如果你使用的
\TeX\ 系统版本较旧,可能没有包含 \pkg{biblatex-gb7714-2015} 宏包,需要手动安装。
\BibLaTeX\ 与 \pkg{biblatex-gb7714-2015} 目前在活跃地更新,为避免一些兼容性问
题,推荐使用较新的版本。

正文中引用参考文献时,使用 \verb|\cite{key1,key2,key3...}| 可以产生“上标引用的
参考文献”,如 \cite{Meta_CN,chen2007act,DPMG}。使用
\verb|\parencite{key1,key2,key3...}| 则可以产生水平引用的参考文献,例如
\parencite{JohnD,zhubajie,IEEE-1363}。请看下面的例子,将会穿插使用水平的和上标的
参考文献:关于书的\parencite{Meta_CN,JohnD,IEEE-1363},关于期刊的
\cite{chen2007act,chen2007ewi},会议论文 \parencite{DPMG,kocher99,cnproceed},硕
士学位论文\parencite{zhubajie,metamori2004},博士学位论文
\cite{shaheshang,FistSystem01,bai2008},标准文件 \parencite{IEEE-1363},技术报告
\cite{NPB2},电子文献 \parencite{xiaoyu2001, CHRISTINE1998},用户手册
\parencite{RManual}。

当需要将参考文献条目加入到文献表中但又不在正文中引用,可以使用
\verb|\nocite{key1,key2,key3...}|。使用 \verb|\nocite{*}| 可以将参考文献数据库中
的所有条目加入到文献表中。

%# -*- coding: utf-8-unix -*-
%%==================================================
%% conclusion.tex for SJTUThesis
%% Encoding: UTF-8
%%==================================================

\begin{summary}

这里是全文总结内容。

2015年2月28日,中央在北京召开全国精神文明建设工作表彰暨学雷锋志愿服务大会,公布全国文明城市(区)、文明村镇、文明单位名单。上海交通大学荣获全国文明单位称号。         

全国文明单位这一荣誉是对交大人始终高度重视文明文化工作的肯定,是对交大长期以来文明创建工作成绩的褒奖。在学校党委、文明委的领导下,交大坚持将文明创建工作纳入学校建设世界一流大学的工作中,全体师生医护员工群策群力、积极开拓,落实国家和上海市有关文明创建的各项要求,以改革创新、科学发展为主线,以质量提升为目标,聚焦文明创建工作出现的重点和难点,优化文明创建工作机制,传播学校良好形象,提升社会美誉度,显著增强学校软实力。2007至2012年间,上海交大连续三届荣获“上海市文明单位”称号,成为创建全国文明单位的新起点。         

上海交大自启动争创全国文明单位工作以来,凝魂聚气、改革创新,积极培育和践行社会主义核心价值观。坚持统筹兼顾、多措并举,将争创全国文明单位与学校各项中心工作紧密结合,着力构建学校文明创建新格局,不断提升师生医护员工文明素养,以“冲击世界一流大学汇聚强大精神动力”为指导思想,以“聚焦改革、多元推进、以评促建、丰富内涵、彰显特色”为工作原则,并由全体校领导群策领衔“党的建设深化、思想教育深入、办学成绩显著、大学文化丰富、校园环境优化、社会责任担当”六大板块共28项重点突破工作,全面展现近年来交大文明创建工作的全貌和成就。         

进入新阶段,学校将继续开拓文明创建工作新格局,不断深化工作理念和工作实践,创新工作载体、丰富活动内涵、凸显创建成效,积极服务于学校各项中心工作和改革发展的大局面,在上级党委、文明委的关心下,在学校党委的直接领导下,与时俱进、开拓创新,为深化内涵建设、加快建成世界一流大学、推动国家进步和社会发展而努力奋斗!       

上海交通大学医学院附属仁济医院也获得全国文明单位称号。      

\end{summary}


% 使用英文字母对附录编号
\appendix

% 附录内容,本科学位论文可以用翻译的文献替代。
% !TEX root = ../thesis.tex

\chapter{Maxwell Equations}

选择二维情况,有如下的偏振矢量:
\begin{subequations}
  \begin{align}
    {\bf E} &= E_z(r, \theta) \hat{\bf z}, \\
    {\bf H} &= H_r(r, \theta) \hat{\bf r} + H_\theta(r, \theta) \hat{\bm\theta}.
  \end{align}
\end{subequations}
对上式求旋度:
\begin{subequations}
  \begin{align}
    \nabla \times {\bf E} &= \frac{1}{r} \frac{\partial E_z}{\partial\theta}
      \hat{\bf r} - \frac{\partial E_z}{\partial r} \hat{\bm\theta}, \\
    \nabla \times {\bf H} &= \left[\frac{1}{r} \frac{\partial}{\partial r}
      (r H_\theta) - \frac{1}{r} \frac{\partial H_r}{\partial\theta} \right]
      \hat{\bf z}.
  \end{align}
\end{subequations}
因为在柱坐标系下,$\overline{\overline\mu}$ 是对角的,所以 Maxwell 方程组中电场
$\bf E$ 的旋度:
\begin{subequations}
  \begin{align}
    & \nabla \times {\bf E} = \upi \omega {\bf B}, \\
    & \frac{1}{r} \frac{\partial E_z}{\partial\theta} \hat{\bf r} -
      \frac{\partial E_z}{\partial r}\hat{\bm\theta} = \upi \omega \mu_r H_r
      \hat{\bf r} + \upi \omega \mu_\theta H_\theta \hat{\bm\theta}.
  \end{align}
\end{subequations}
所以 $\bf H$ 的各个分量可以写为:
\begin{subequations}
  \begin{align}
    H_r &= \frac{1}{\upi \omega \mu_r} \frac{1}{r}
      \frac{\partial E_z}{\partial\theta}, \\
    H_\theta &= -\frac{1}{\upi \omega \mu_\theta}
      \frac{\partial E_z}{\partial r}.
  \end{align}
\end{subequations}
同样地,在柱坐标系下,$\overline{\overline\epsilon}$ 是对角的,所以 Maxwell 方程
组中磁场 $\bf H$ 的旋度:
\begin{subequations}
  \begin{align}
    & \nabla \times {\bf H} = -\upi \omega {\bf D}, \\
    & \left[\frac{1}{r} \frac{\partial}{\partial r}(r H_\theta) - \frac{1}{r}
      \frac{\partial H_r}{\partial\theta} \right] \hat{\bf z} = -\upi \omega
      {\overline{\overline\epsilon}} {\bf E} = -\upi \omega \epsilon_z E_z
      \hat{\bf z}, \\
    & \frac{1}{r} \frac{\partial}{\partial r}(r H_\theta) - \frac{1}{r}
      \frac{\partial H_r}{\partial\theta} = -\upi \omega \epsilon_z E_z.
  \end{align}
\end{subequations}
由此我们可以得到关于 $E_z$ 的波函数方程:
\begin{equation}
  \frac{1}{\mu_\theta \epsilon_z} \frac{1}{r} \frac{\partial}{\partial r}
  \left(r \frac{\partial E_z}{\partial r} \right) + \frac{1}{\mu_r \epsilon_z}
  \frac{1}{r^2} \frac{\partial^2E_z}{\partial\theta^2} +\omega^2 E_z = 0.
\end{equation}

% !TEX root = ../thesis.tex

\chapter{绘制流程图}

图~\ref{fig:flow_chart} 是一张流程图示意。使用 \pkg{tikz} 环境,搭配四种预定义节
点(\verb+startstop+、\verb+process+、\verb+decision+和\verb+io+),可以容易地绘
制出流程图。

\begin{figure}[!htp]
  \centering
  \resizebox{6cm}{!}{\begin{tikzpicture}[node distance=2cm]
    \node (pic) [startstop] {待测图片};
    \node (bg) [io, below of=pic] {读取背景};
    \node (pair) [process, below of=bg] {匹配特征点对};
    \node (threshold) [decision, below of=pair, yshift=-0.5cm] {多于阈值};
    \node (clear) [decision, right of=threshold, xshift=3cm] {清晰?};
    \node (capture) [process, right of=pair, xshift=3cm, yshift=0.5cm] {重采};
    \node (matrix_p) [process, below of=threshold, yshift=-0.8cm] {透视变换矩阵};
    \node (matrix_a) [process, right of=matrix_p, xshift=3cm] {仿射变换矩阵};
    \node (reg) [process, below of=matrix_p] {图像修正};
    \node (return) [startstop, below of=reg] {配准结果};
     
    %连接具体形状
    \draw [arrow](pic) -- (bg);
    \draw [arrow](bg) -- (pair);
    \draw [arrow](pair) -- (threshold);

    \draw [arrow](threshold) -- node[anchor=south] {否} (clear);

    \draw [arrow](clear) -- node[anchor=west] {否} (capture);
    \draw [arrow](capture) |- (pic);
    \draw [arrow](clear) -- node[anchor=west] {是} (matrix_a);
    \draw [arrow](matrix_a) |- (reg);

    \draw [arrow](threshold) -- node[anchor=east] {是} (matrix_p);
    \draw [arrow](matrix_p) -- (reg);
    \draw [arrow](reg) -- (return);
\end{tikzpicture}
}
  \bicaption{绘制流程图效果}{Flow chart}
  \label{fig:flow_chart}
\end{figure}


% 文后无编号部分
\backmatter

% 参考资料
\printbibliography[heading=bibintoc]

% 用于盲审的论文需隐去致谢、发表论文、参与项目、申请专利、简历

% 致谢
% !TEX root = ../main.tex

\begin{acknowledgements}
  感谢那位最先制作出博士学位论文 \LaTeX{} 模板的物理系同学!

  感谢 William Wang 同学对模板移植做出的贡献!

  感谢 \href{https://github.com/weijianwen}{@weijianwen} 学长开创性的工作!

  感谢 \href{https://github.com/sjtug}{@sjtug} 对 0.10 及之后版本的开发和维护工作!

  感谢所有为模板贡献过代码的\href{https://github.com/sjtug/SJTUThesis/graphs/contributors}{同学们}, 以及所有测试和使用模板的各位同学!

  感谢 \LaTeX 和 \href{https://github.com/sjtug/SJTUThesis}{SJTUThesis},帮我节省了不少时间。
\end{acknowledgements}


% 发表论文、参与项目、申请专利、简历
% 盲审论文中,发表学术论文及参与科研情况等仅以第几作者注明即可,不要出现作者或他人姓名
% !TEX root = ../thesis.tex

%TC:ignore

\begin{publications}
  \item Chen H, Chan C~T. Acoustic cloaking in three dimensions using acoustic metamaterials[J]. Applied Physics Letters, 2007, 91:183518.
  \item Chen H, Wu B~I, Zhang B, et al. Electromagnetic Wave Interactions with a Metamaterial Cloak[J]. Physical Review Letters, 2007, 99(6):63903.
\end{publications}

\begin{publications*}
  \item 第一作者. 中文核心期刊论文, 2007.
  \item 第一作者. EI 国际会议论文, 2006.
\end{publications*}

%TC:endignore

% !TEX root = ../thesis.tex

%TC:ignore

\begin{projects}
  \item 参与973项目子课题(2007年6月--2008年5月)
  \item 参与自然基金项目(2005年5月--2005年8月)
  \item 参与国防项目(2005年8月--2005年10月)
\end{projects}

\begin{projects*}
  \item 973项目“XXX”
  \item 自然基金项目“XXX”
  \item 国防项目“XXX”
\end{projects*}

%TC:endignore

%# -*- coding: utf-8-unix -*-
\begin{patents}{99}
    \item 第一发明人,“永动机”,专利申请号202510149890.0
\end{patents}

\begin{resume}
  \begin{resumesection}{基本情况}
    某某,yyyy 年 mm 月生于 xxxx。
  \end{resumesection}

  \begin{resumelist}{教育背景}
    \item yyyy 年 mm 月至今,上海交通大学,博士研究生,xx 专业
    \item yyyy 年 mm 月至 yyyy 年 mm 月,上海交通大学,硕士研究生,xx 专业
    \item yyyy 年 mm 月至 yyyy 年 mm 月,上海交通大学,本科,xx 专业
  \end{resumelist}

  \begin{resumesection}{研究兴趣}
    \LaTeX{} 排版
  \end{resumesection}

  \begin{resumelist}{联系方式}
    \item 地址: 上海市闵行区东川路 800 号,200240
    \item E-mail: \email{xxx@sjtu.edu.cn}
  \end{resumelist}
\end{resume}


% 中文学士学位论文要求在最后有一个英文大摘要,单独编页码,英文学士学位论文不需要
%# -*- coding: utf-8-unix -*-
\begin{bigabstract}
Affronting discretion as do is announcing. Now months esteem oppose nearer enable too six. She numerous unlocked you perceive speedily. Affixed offence spirits or ye of offices between. Real on shot it were four an as. Absolute bachelor rendered six nay you juvenile. Vanity entire an chatty to. 

Admiration we surrounded possession frequently he. Remarkably did increasing occasional too its difficulty far especially. Known tiled but sorry joy balls. Bed sudden manner indeed fat now feebly. Face do with in need of wife paid that be. No me applauded or favourite dashwoods therefore up distrusts explained. 

Is education residence conveying so so. Suppose shyness say ten behaved morning had. Any unsatiable assistance compliment occasional too reasonably advantages. Unpleasing has ask acceptance partiality alteration understood two. Worth no tiled my at house added. Married he hearing am it totally removal. Remove but suffer wanted his lively length. Moonlight two applauded conveying end direction old principle but. Are expenses distance weddings perceive strongly who age domestic. 

Unpleasant astonished an diminution up partiality. Noisy an their of meant. Death means up civil do an offer wound of. Called square an in afraid direct. Resolution diminution conviction so mr at unpleasing simplicity no. No it as breakfast up conveying earnestly immediate principle. Him son disposed produced humoured overcame she bachelor improved. Studied however out wishing but inhabit fortune windows. 

Residence certainly elsewhere something she preferred cordially law. Age his surprise formerly mrs perceive few stanhill moderate. Of in power match on truth worse voice would. Large an it sense shall an match learn. By expect it result silent in formal of. Ask eat questions abilities described elsewhere assurance. Appetite in unlocked advanced breeding position concerns as. Cheerful get shutters yet for repeated screened. An no am cause hopes at three. Prevent behaved fertile he is mistake on. 

Rendered her for put improved concerns his. Ladies bed wisdom theirs mrs men months set. Everything so dispatched as it increasing pianoforte. Hearing now saw perhaps minutes herself his. Of instantly excellent therefore difficult he northward. Joy green but least marry rapid quiet but. Way devonshire introduced expression saw travelling affronting. Her and effects affixed pretend account ten natural. Need eat week even yet that. Incommode delighted he resolving sportsmen do in listening. 

Sex and neglected principle ask rapturous consulted. Object remark lively all did feebly excuse our wooded. Old her object chatty regard vulgar missed. Speaking throwing breeding betrayed children my to. Me marianne no he horrible produced ye. Sufficient unpleasing an insensible motionless if introduced ye. Now give nor both come near many late. 

Is branched in my up strictly remember. Songs but chief has ham widow downs. Genius or so up vanity cannot. Large do tried going about water defer by. Silent son man she wished mother. Distrusts allowance do knowledge eagerness assurance additions to. 

Fat son how smiling mrs natural expense anxious friends. Boy scale enjoy ask abode fanny being son. As material in learning subjects so improved feelings. Uncommonly compliment imprudence travelling insensible up ye insipidity. To up painted delight winding as brandon. Gay regret eat looked warmth easily far should now. Prospect at me wandered on extended wondered thoughts appetite to. Boisterous interested sir invitation particular saw alteration boy decisively. 

Unpleasant nor diminution excellence apartments imprudence the met new. Draw part them he an to he roof only. Music leave say doors him. Tore bred form if sigh case as do. Staying he no looking if do opinion. Sentiments way understood end partiality and his. 

\end{bigabstract}

\end{document}
