% !TeX encoding = UTF-8

% 载入 SJTUThesis 模版
\documentclass[type=master]{sjtuthesis}
% 选项
%   type=[doctor|master|bachelor|course],     % 可选(默认:doctor),论文类型
%   zihao=[-4|5],                             % 可选(研究生默认:-4,本科默认:5),正文字号大小
%   lang=[zh|en],                             % 可选(默认:zh),论文的主要语言
%   review,                                   % 可选(默认:关闭),盲审模式
%   [twoside|oneside]                         % 可选(默认:twoside),单双页模式

% 论文基本配置,加载宏包等全局配置
% !TEX root = ./main.tex

\sjtusetup{
  %
  %******************************
  % 注意:
  %   1. 配置里面不要出现空行
  %   2. 不需要的配置信息可以删除
  %******************************
  %
  % 信息录入
  %
  info = {%
    %
    % 标题
    %
    zh / title           = {上海交通大学学位论文 \LaTeX{} 模板示例文档},
    en / title           = {A Sample Document for \LaTeX-based SJTU Thesis Template},
    %
    % 标题页标题
    %   可使用“\\”命令手动控制换行
    %
    % zh / display-title   = {上海交通大学学位论文\\ \LaTeX{} 模板示例文档},
    % en / display-title   = {A Sample Document \\ for \LaTeX-based SJTU Thesis Template},
    %
    % 关键词
    %
    zh / keywords        = {上海交大, 饮水思源, 爱国荣校},
    en / keywords        = {SJTU, master thesis, XeTeX/LaTeX template},
    %
    % 姓名
    %
    zh / author          = {某\quad{}某},
    en / author          = {Mo Mo},
    %
    % 指导教师
    %
    zh / supervisor      = {某某教授},
    en / supervisor      = {Prof. Mou Mou},
    %
    % 副指导教师
    %
    % assoc-supervisor  = {某某教授},
    % assoc-supervisor* = {Prof. Uom Uom},
    %
    % 学号
    %
    id              = {0010900990},
    %
    % 学位
    %   本科生不需要填写
    %
    zh / degree          = {工学硕士},
    en / degree          = {Master of Engineering},
    %
    % 专业
    %
    zh / major           = {某某专业},
    en / major           = {A Very Important Major},
    %
    % 所属院系
    %
    zh / department      = {某某系},
    en / department      = {Depart of XXX},
    %
    % 答辩日期
    %   使用 ISO 格式 (yyyy-mm-dd);默认为当前时间
    %
    % date            = {2014-12-17},
    %
    % 资助基金
    %
    % zh / fund  = {
    %                {国家 973 项目 (No. 2025CB000000)},
    %                {国家自然科学基金 (No. 81120250000)},
    %              },
    % en / fund  = {
    %                {National Basic Research Program of China (Grant No. 2025CB000000)},
    %                {National Natural Science Foundation of China (Grant No. 81120250000)},
    %              },
  },
  %
  % 风格设置
  %
  style = {%
    %
    % 论文标题页 logo 颜色 (red/blue/black)
    %
    % title-logo-color = black,
  },
  %
  % 名称设置
  %
  name = {
    % bib             = {References},
    % ack             = {谢\hspace{\ccwd}辞},
    % achv            = {攻读学位期间完成的论文},
  },
}

% 使用 BibLaTeX 处理参考文献
%   biblatex-gb7714-2015 常用选项
%     gbnamefmt=lowercase     姓名大小写由输入信息确定
%     gbpub=false             禁用出版信息缺失处理
\usepackage[backend=biber,style=gb7714-2015]{biblatex}
% 文献表字体
% \renewcommand{\bibfont}{\zihao{-5}}
% 文献表条目间的间距
\setlength{\bibitemsep}{0pt}
% 导入参考文献数据库
\addbibresource{bibdata/thesis.bib}

% 脚注格式
\usepackage[perpage,bottom,hang]{footmisc}

% 定义图片文件目录与扩展名
\graphicspath{{figures/}}
\DeclareGraphicsExtensions{.pdf,.eps,.png,.jpg,.jpeg}

% 确定浮动对象的位置,可以使用 [H],强制将浮动对象放到这里(可能效果很差)
% \usepackage{float}

% 固定宽度的表格
% \usepackage{tabularx}

% 使用三线表:toprule,midrule,bottomrule。
\usepackage{booktabs}

% 表格中支持跨行
\usepackage{multirow}

% 表格中数字按小数点对齐
\usepackage{dcolumn}
\newcolumntype{d}[1]{D{.}{.}{#1}}

% 使用长表格
\usepackage{longtable}

% 附带脚注的表格
\usepackage{threeparttable}

% 附带脚注的长表格
\usepackage{threeparttablex}

% 算法环境宏包
\usepackage[ruled,vlined,linesnumbered]{algorithm2e}
% \usepackage{algorithm, algorithmicx, algpseudocode}

% 代码环境宏包
\usepackage{listings}
\lstnewenvironment{codeblock}[1][]%
  {\lstset{style=lstStyleCode,#1}}{}

% 代码高亮宏包(需要 PATH 中包含 pygmentize 命令,且构建环境允许添加 -shell-escape 参数)
% \usepackage{minted}

% 物理科学和技术中使用的数学符号,定义了 \qty 命令,与 siunitx 3.0 有冲突
% \usepackage{physics}

% 直立体数学符号
\providecommand{\dd}{\mathop{}\!\mathrm{d}}
\providecommand{\ee}{\mathrm{e}}
\providecommand{\ii}{\mathrm{i}}
\providecommand{\jj}{\mathrm{j}}

% 国际单位制宏包
\usepackage{siunitx}[=v2]

% 定理环境宏包
\usepackage{ntheorem}
% \usepackage{amsthm}

% 绘图宏包
\usepackage{tikz}
\usetikzlibrary{shapes.geometric, arrows}

% 一些文档中用到的 logo
\usepackage{hologo}
\providecommand{\XeTeX}{\hologo{XeTeX}}
\providecommand{\BibLaTeX}{\textsc{Bib}\LaTeX}

% 借用 ltxdoc 里面的几个命令方便写文档
\DeclareRobustCommand\cs[1]{\texttt{\char`\\#1}}
\providecommand\pkg[1]{{\sffamily#1}}

% 自定义命令

% E-mail
\newcommand{\email}[1]{\href{mailto:#1}{\texttt{#1}}}

% hyperref 宏包在最后调用
\usepackage{hyperref}

% 自动引用题注更正为中文
\def\equationautorefname{式}
\def\footnoteautorefname{脚注}
\def\itemautorefname{项}
\def\figureautorefname{图}
\def\tableautorefname{表}
\def\partautorefname{篇}
\def\appendixautorefname{附录}
\def\chapterautorefname{章}
\def\sectionautorefname{节}
\def\subsectionautorefname{小节}
\def\subsubsectionautorefname{小节}
\def\paragraphautorefname{段落}
\def\subparagraphautorefname{子段落}
\def\FancyVerbLineautorefname{行}
\def\theoremautorefname{定理}
\def\listingautorefname{代码清单}


\begin{document}

%TC:ignore

% 标题页
\maketitle

% 原创性声明及使用授权书
\copyrightpage
% 插入外置原创性声明及使用授权书
% \copyrightpage[scans/sample-copyright-old.pdf]

% 前置部分
\frontmatter

% 摘要
% !TEX root = ../main.tex

\begin{abstract}[zh]
  中文摘要应该将学位论文的内容要点简短明了地表达出来,应该包含论文中的基本信息,
  体现科研工作的核心思想。摘要内容应涉及本项科研工作的目的和意义、研究方法、研究
  成果、结论及意义。注意突出学位论文中具有创新性的成果和新见解的部分。摘要中不宜
  使用公式、化学结构式、图表和非公知公用的符号和术语,不标注引用文献编号。硕士学
  位论文中文摘要字数为 500 字左右,博士学位论文中文摘要字数为 800 字左右。英文摘
  要内容应与中文摘要内容一致。

  摘要页的下方注明本文的关键词(4 \textasciitilde{} 6 个)。
\end{abstract}

\begin{abstract}[en]
  Shanghai Jiao Tong University (SJTU) is a key university in China. SJTU was
  founded in 1896. It is one of the oldest universities in China. The University
  has nurtured large numbers of outstanding figures include JIANG Zemin, DING
  Guangen, QIAN Xuesen, Wu Wenjun, WANG An, etc.

  SJTU has beautiful campuses, Bao Zhaolong Library, Various laboratories. It
  has been actively involved in international academic exchange programs. It is
  the center of CERNet in east China region, through computer networks, SJTU has
  faster and closer connection with the world.
\end{abstract}


% 目录
\tableofcontents
% 插图索引
\listoffigures*
% 表格索引
\listoftables*
% 算法索引
\listofalgorithms*

% 符号对照表
% !TEX root = ../main.tex

\begin{nomenclature}
\label{chap:symb}

\begin{longtable}{rl}
  $\epsilon$  & 介电常数  \\  
  $\mu$       & 磁导率    \\
  $\epsilon$  & 介电常数  \\
  $\mu$       & 磁导率    \\
  $\epsilon$  & 介电常数  \\
  $\mu$       & 磁导率    \\
  $\epsilon$  & 介电常数  \\
  $\mu$       & 磁导率    \\
  $\epsilon$  & 介电常数  \\
  $\mu$       & 磁导率    \\
  $\epsilon$  & 介电常数  \\
  $\mu$       & 磁导率    \\
  $\epsilon$  & 介电常数  \\
  $\mu$       & 磁导率    \\
  $\epsilon$  & 介电常数  \\
  $\mu$       & 磁导率    \\
  $\epsilon$  & 介电常数  \\
  $\mu$       & 磁导率    \\
  $\epsilon$  & 介电常数  \\
  $\mu$       & 磁导率    \\
  $\epsilon$  & 介电常数  \\
  $\mu$       & 磁导率    \\
  $\epsilon$  & 介电常数  \\
  $\mu$       & 磁导率    \\
  $\epsilon$  & 介电常数  \\
  $\mu$       & 磁导率    \\
  $\epsilon$  & 介电常数  \\
  $\mu$       & 磁导率    \\
  $\epsilon$  & 介电常数  \\
  $\mu$       & 磁导率    \\
  $\epsilon$  & 介电常数  \\
  $\mu$       & 磁导率    \\
  $\epsilon$  & 介电常数  \\
  $\mu$       & 磁导率    \\
  $\epsilon$  & 介电常数  \\
  $\mu$       & 磁导率    \\
  $\epsilon$  & 介电常数  \\
  $\mu$       & 磁导率    \\
  $\epsilon$  & 介电常数  \\
  $\mu$       & 磁导率    \\
  $\epsilon$  & 介电常数  \\
  $\mu$       & 磁导率    \\
  $\epsilon$  & 介电常数  \\
  $\mu$       & 磁导率    \\
  $\epsilon$  & 介电常数  \\
  $\mu$       & 磁导率    \\
  $\epsilon$  & 介电常数  \\
  $\mu$       & 磁导率    \\
  $\epsilon$  & 介电常数  \\
  $\mu$       & 磁导率    \\
  $\epsilon$  & 介电常数  \\
  $\mu$       & 磁导率    \\
  $\epsilon$  & 介电常数  \\
  $\mu$       & 磁导率    \\
\end{longtable}

\end{nomenclature}


%TC:endignore

% 主体部分
\mainmatter

% 正文内容
% !TEX root = ../main.tex

\chapter{绪论}


\section{研究背景与意义}

命名实体识别指的是从文本中识别出重要对象(例如人,组织,地点)的自然语言处理任务\parencite{erdogan2010sequence}。
后面我们将使用NER来指代命名实体识别和分类。NER是信息提取、问答系统、句法分析、机器翻译、
面向语义网的元数据标注等应用领域的重要基础工具,在自然语言处理技术走向实用
化的过程中占有重要地位。

NER任务通常包括两部分:(1)实体边界识别;(2)确定实体类别(人名、地名、机构名或其他)。
英语中的命名实体具有比较明显的形式标志(即实体中的每个词的第一个字母要大写),所以实体边界
识别相对容易,任务的重点是确定实体的类别。和英语相比,汉语命名实体识别任务更加复杂,而且相
对于实体类别标注子任务,实体边界的识别更加困难。


\section{国内外研究进展}


\subsection{基于规则的命名实体识别方法}

基于规则和字典的方法是最初代的命名实体识别使用的方法,这些方法多采用由语言学家通过人工方式,
依据数据集特征构建的特定规则模板或者特殊词典。规则包括关键词、位置词、方位词、中心词、指示词、
统计信息、标点符号等。词典是由特征词构成的词典和外部词典共同组成,外部词典指已有的常识词典。
制定好规则和词典后,通常使用匹配的方式对文本进行处理以实现命名实体识别。

这种方法所训练出来的模型通常过拟合于非常特定的结构化文本语料库,如军事信息集、海军作战报告、并购新闻等。


\subsection{基于统计的命名实体识别方法}

\subsubsection{监督学习方法}

在大型注释语料库上训练的监督学习技术为NER提供了当时最好的结果。著名的监督学习算法
主要包括隐马尔可夫模型(HMM)、决策树、最大熵模型(ME)、支持向量机(SVM)、条件随机场(CRF)。
特别是CRF算法是最有效的NER算法之一。NER需要使用许多前导和滞后的非局部序列训练输出标签的概率,
这使得像CRF这样的判别模型比生成模型更为适合。尽管ME模型放宽了生成模型所做的强独立性假设,但他们存在标签偏差问题,
即模型偏向于输出过渡较少的状态。CRF通过联合考虑所有状态中不同特征的权重来解决这一问题,而不是将状态级别的转移概率归一化。
\parencite{sarawagi2004semi}通过制定半CRF模型改进了现有CRF模型,该模型将标签分配给子序列而不是单个实体,并且没有任何显著的额外计算复杂度。
\parencite{passos2014lexicon}在公共数据上使用他们的词典注入Skip-gram模型来学习高质量的短语向量,并将其插入对数线性CRF系统。
实体分析的联合模型(多任务模型)在单个任务上的结果比单独为NER优化的模型更好。
\parencite{durrett2014joint}开发了一个结构化的CRF模型,该模型通过训练3个任务来提高NER(以及其他实体分析任务)的性能,
例如共指解析(文档内聚类)、命名实体识别(粗语义类型)和实体链接(匹配维基百科实体)。



\subsubsection{半监督学习方法}

\parencite{luo2015joint}提出了JERL(Joint Entity Recognition and Linking)模型,即扩展半CRF模型以捕获NER和实体链接之间的依赖关系。
监督学习方法遇到了障碍,因为可用于学习判别特征的结构化文本是有限的。这催生了半监督学习方法,
它利用了非结构化文本的指数增长量,换句话说,网页以无监督的方式从种子注释语料库中获取上下文信息,即bootstarpping。
\parencite{luo2015joint}提出了一种无监督方法,从大规模未标记数据中创建资源丰富的浓缩特征表示,以有效地训练有监督的NER系统,同时仍保持现有高维半监督解决方案的最新结果。



\subsubsection{无监督学习方法}

无监督学习方法主要是从作为输入词的上下文中生成附加特征以与其他 NER 方法结合使用的手段。
\parencite{lin2009phrase}通过在搜索引擎查询日志的私有数据库上执行 K-means 聚类以使用聚类特征来训练他们的CRF模型,
从而在不使用地名词典的情况下获得了当时最优的结果。无监督学习技术工业应用的一个例外是使用词汇资源的聚类方法,
例如Wordnet以便从Wordnet本身分配命名实体类型。尽管为此目的广泛采用了CRF,但仍有许多关于使用神经网络进行NER的论文。
\parencite{ratinov2009design}在感知机模型中使用了非局部特征、从维基百科中提取的地名词典和类似布朗簇的词表示。由于多层前馈神经网络是通用逼近器,
因此这种神经网络也可以潜在地解决NER任务。


\subsection{基于深度学习的命名实体识别方法}

\subsubsection{卷积神经网络}

卷积神经网络(CNN)可以使用反向传播学习丛单词中提取特征向量表示。因此,在从可变长度的输入序列中提取高阶特征时,
CNN似乎是自然选择。有两种方法:(1)窗口法和(2)(卷积)句法。窗口方法假设分配给单个单词的标签取决于它的上下文
(即出现在给定单词之前或之后的单词),该方法更适合于像NER这样的序列标记任务。

\begin{equation}
	f^{l}_{\theta}
	= \langle LT_W( [ w ] ^T_l) \rangle ^{d_{win}}_t
	= \left(
			\begin{array}{c}
			\langle W \rangle ^l_{[w]_{t - d_{win} / 2}} \\
			\vdots \\
			\langle W \rangle ^l_{[w]_t} \\
			\vdots \\
			\langle W \rangle ^l_{[w]_{t + d_{win} / 2}}
		\end{array} 
	\right)
\end{equation}


\begin{equation}
	f^{l}_{\theta} 
	= W^l f^{l - 1}_{\theta} + b^l
\end{equation}

这里的$W^l \in R^{ n^l_h \times n^{l - 1}_h}$和$b^l \in R^{n^l_h}$是使用反向传播训练的,
$n^l_h$是一个超参数,表示该层中隐藏单元的数量。固定大小的矢量输入可以通过如上所示的多个线性变换层进行传递。
为了从输入序列中捕获高阶特征,存在“硬”双曲正切函数层。“硬”双曲正切在防止过拟合的同时,与精确双曲正切函数相比,
计算效率更高。窗口特性的一个警告是,句首和句尾单词的上下文没有明确定义。因此,在输入句子的开头和结尾都有相当于
窗口一半大小的填充词,类似于序列模型中的开始和停止指示器。


\begin{equation}
	[f^{l}_{\theta}]_i 
	= Hard tanh([f^{l - 1}_{\theta}]_i)
\end{equation}


\begin{equation}
	Hard tanh(x)
    = \begin{cases}
        -1 & x < -1 \\
        x & -1 \geq x \leq 1 \\
        1 & x \geq 1
    \end{cases}
\end{equation}


使用CNN进行句子建模可以追溯到\parencite{collobert2008unified}。该工作使用多任务学习来输出对NLP任务的多个预测,
如POS标记、块、命名实体标记、语义角色、语义相似词和一个语言模型。使用一个查找表将每个单词转换为用户定义的维度向量。
因此,通过对每个单词应用查找表,n个单词的输入序列$s_1,s_2,\cdots, s_n$被转换为一系列向量$ws_1,ws_2, \cdots, ws_n$。


这可以看作是一种原始词嵌入方法,其权值是在网络训练中习得的。在\parencite{collobert2011natural}中,Collobert扩展了他们的工作,
提出了一个通用的基于CNN的框架来解决过多的NLP任务。这两篇文章使得CNN在自然语言处理研究人员中得到了广泛普及。
鉴于CNN在计算机视觉任务上已经展现出了它们的实力,人们更容易相信它们的表现。
CNN能够从输入句子中提取显著的N-gram特征,为下游任务创建句子的信息潜在语义表示。
这样的应用是由\parencite{collobert2011natural, kim2016character, kalchbrenner2014convolutional}初创的,这催生了后续文献中以CNN为基础的网络的大量扩散。

\subsubsection{循环神经网络}


循环神经网络是深度神经网络体系结构的另一种形式。一般来说,CNN被用来表示位置不变的函数(如词袋),
而RNN则表示顺序的体系结构(如句子、段落等)。很明显,相对于分层CNN, RNN更适合于像NER这样的序列建模任务。
虽然我们看到窗口方法有助于处理CNN中的序列输入,但除了在交叉验证中固定的之外,没有办法对上下文依赖性建模。
RNN帮助捕捉单词或句子的依赖性,这超出了CNN的固定范围。简单的RNN没有门控机制。RNN是一种跨时间展开的网络,
因此提供了记忆的空间表示。对于一个给定的输入,RNN计算一个隐藏状态如下:

\begin{equation}
	[s^t_{\theta, l}]_i = g_{\theta, l}([f^t_{\theta, l}]_i)
\end{equation}

\begin{equation}
	[f^t_{\theta, l}]_i = W^l[x^t_l]_i + [s^{t-1}_{\theta, l}]_i
\end{equation}

这里$[s^t_{\theta, l}]_i$是输入序列的单位在$t$时刻和$l$层的隐藏状态。
$g_{\theta, l}$是在以$[f^t_{\theta, l}]_i$为输入层上的一个非线性函数(如$tanh$等)。
$x^t_l$是在$t$时刻和通过训练得到的权重$W^l \in R^{d_{s^{t}_{\theta, l}} \times d_{x^t_l}}$上输入序列的第$i$个单元(单词,句子等)。
简单RNN中的隐藏状态可以看作是它的记忆分量。然而,简单RNN存在梯度消失的问题,这使得反向传播很难学习到之前的权值。
因此,简单RNN增加了门控机制来克服收敛问题。具有门控机制的RNN变体在神经网络中最受欢迎的是长短期记忆网络(LSTM)和门控循环单元(GRUs)。


\paragraph{长短期记忆网络}

LSTM\parencite{graves2012long}的新奇之处在于它能够跨越长时间间隔(快速学习许多时间步后缓慢变化的权值,如长期记忆)
以及保存最近的输入(例如短期记忆)。此外,LSTM架构确保了不断的重新加权,从而避免了通过隐藏状态错误流的爆发或消失。

\begin{numcases}{}
	i_t = \sigma(x_t U^i + h_{t-1} W^i + b_i) \\
	f_t = \sigma(x_t U^f + h_{t-1} W^f + b_f) \\
	o_t = \sigma(x_t U^o + h_{t-1} W^o + b_o) \\
	q_t = tanh(x_t U^q + h_{t-1} W^q + b_q) \\
	p_t = f_t \times p_{t-1} + i_t \times q_t \\
	h_t = o_t \times tanh(p_t)
\end{numcases}

LSTM有三个门:输入门$i_t$、遗忘门$f_t$和输出门$o_t$,它们是$Sigmoid$函数在输入$x_t$和前面的隐藏状态$h_{t-1}$上的输出。
为了生成当前第$t$步的隐藏状态,它首先通过在输入$x_t$和之前的隐藏状态$h_{t-1}$上运行非线性$tanh$函数生成一个临时变量$q_t$。
然后LSTM计算一个在时刻$t$更新的历史变量$p_t$作为前一个历史状态$p_{t-1}$和当前临时变量$q_t$的线性组合,
分别由当前遗忘门$f_t$和当前输入门$o_t$。最后,LSTM运行$tanh$,并根据当前输出门对其进行缩放,以获得更新后的隐藏状态。

虽然LSTM善于近似序列的当前单元与前一个单元的相关性,但它没有考虑到当前单元与序列中其右侧单元的相关性。
\parencite{lample2016neural}通过实现双向LSTM解决了这个问题\parencite{graves2005framewise}。
换句话说,有两个独立的LSTM,分别使用目标词的左侧和右侧的输入序列片段进行训练。
该模型通过分别连接前向和后向LSTM的左右上下文表示,即隐藏状态$h_t = [h_t^{\rightarrow};h_t^{\leftarrow} ]$,获得目标词的完整表示。


\paragraph{门控循环单元}

与LSTM相比,门控循环单元[57]是一种较新的、较不复杂的RNN变体。

GRU与LSTM相似,因为它可以调节错误流,从而避免梯度消失\parencite{bengio1994learning}。然而,GRU与LSTM有许多关键的区别。
GRN不像LSTM那样有独立的存储单元。因此,GRN缺乏对内存内容(即LSTM的输出门)的受控外显。
与LSTM不同,GRN在没有任何控制的情况下公开完整的内容。此外,GRN控制着从前一个激活状态到当前候选激活状态的信息流。
另一方面,LSTM在不控制历史信息流的情况下计算最新的内存内容。GRU的工作原理如下:

\begin{numcases}{}
	h^j_t = (1 - z^j_t)h^j_{t-1} + z^j_t \overline{h}^j_t \\
	z^j_t = \sigma(W_z x_t + U_z h_{t-1}) \\
	\overline{h}^j_t = tanh(Wx_t + U(r_t \odot h_{t-1})) \\
	r^j_t = \sigma(W_r x_t + U_r h_{t-1})
\end{numcases}


$h^j_t$是GRU在时刻$t$时第$j$级的激活状态。$\overline{h}^j_t$是GRU在时刻$t$时第$j$级的候选激活状态\parencite{bahdanau2014neural}。
更新门$z^j_t$决定了单元更新其激活状态的程度。重置门$r^j_t$的计算方法与更新门类似。


\section{论文组织结构}


% 上海交通大学是我国历史最悠久的高等学府之一,是教育部直属、教育部与上海市共建的全
% 国重点大学,是国家“七五”、“八五”重点建设和“211 工程”、“985 工程”的首批建
% 设高校。经过 115 年的不懈努力,上海交通大学已经成为一所“综合性、研究型、国际化”
% 的国内一流、国际知名大学,并正在向世界一流大学稳步迈进。 

% {\songti 十九世纪末,甲午战败,民族危难。中国近代著名实业家、教育家盛宣怀和一批
%   有识之士秉持“自强首在储才,储才必先兴学”的信念,于 1896 年在上海创办了交通大
%   学的前身——南洋公学。建校伊始,学校即坚持“求实学,务实业”的宗旨,以培养“第
%   一等人才”为教育目标,精勤进取,笃行不倦,在二十世纪二三十年代已成为国内著名的
%   高等学府,被誉为“东方MIT”。抗战时期,广大师生历尽艰难,移转租界,内迁重庆,
%   坚持办学,不少学生投笔从戎,浴血沙场。解放前夕,广大师生积极投身民主革命,学校
%   被誉为“民主堡垒”。
  
%   新中国成立初期,为配合国家经济建设的需要,学校调整出相当一部分优势专业、师资设
%   备,支持国内兄弟院校的发展。五十年代中期,学校又响应国家建设大西北的号召,根据
%   国务院决定,部分迁往西安,分为交通大学上海部分和西安部分。1959 年 3月两部分同
%   时被列为全国重点大学,7 月经国务院批准分别独立建制,交通大学上海部分启用“上海
%   交通大学”校名。历经西迁、两地办学、独立办学等变迁,为构建新中国的高等教育体
%   系,促进社会主义建设做出了重要贡献。六七十年代,学校先后归属国防科工委和六机部
%   领导,积极投身国防人才培养和国防科研,为“两弹一星”和国防现代化做出了巨大贡
%   献。}

% {\heiti 改革开放以来,学校以“敢为天下先”的精神,大胆推进改革:率先组成教授代
%   表团访问美国,率先实行校内管理体制改革,率先接受海外友人巨资捐赠等,有力地推动
%   了学校的教学科研改革。1984 年,邓小平同志亲切接见了学校领导和师生代表,对学校
%   的各项改革给予了充分肯定。在国家和上海市的大力支持下,学校以“上水平、创一流”
%   为目标,以学科建设为龙头,先后恢复和兴建了理科、管理学科、生命学科、法学和人文
%   学科等。1999 年,上海农学院并入;2005 年,与上海第二医科大学强强合并。至此,学
%   校完成了综合性大学的学科布局。近年来,通过国家“985 工程”和“211 工程”的建
%   设,学校高层次人才日渐汇聚,科研实力快速提升,实现了向研究型大学的转变。与此同
%   时,学校通过与美国密西根大学等世界一流大学的合作办学,实施国际化战略取得重要突
%   破。1985 年开始闵行校区建设,历经 20 多年,已基本建设成设施完善,环境优美的现
%   代化大学校园,并已完成了办学重心向闵行校区的转移。学校现有徐汇、闵行、法华、七
%   宝和重庆南路(卢湾)5 个校区,总占地面积 4840 亩。通过一系列的改革和建设,学校
%   的各项办学指标大幅度上升,实现了跨越式发展,整体实力显著增强,为建设世界一流大
%   学奠定了坚实的基础。}

% {\ifcsname fangsong\endcsname\fangsong\else[无 \cs{fangsong} 字体。]\fi 交通大学
%   始终把人才培养作为办学的根本任务。一百多年来,学校为国家和社会培养了 20余万各
%   类优秀人才,包括一批杰出的政治家、科学家、社会活动家、实业家、工程技术专家和医
%   学专家,如江泽民、陆定一、丁关根、汪道涵、钱学森、吴文俊、徐光宪、张光斗、黄炎
%   培、邵力子、李叔同、蔡锷、邹韬奋、陈敏章、王振义、陈竺等。在中国科学院、中国工
%   程院院士中,有 200 余位交大校友;在国家 23 位“两弹一星”功臣中,有 6 位交大校
%   友;在 18 位国家最高科学技术奖获得者中,有 3 位来自交大。交大创造了中国近现代
%   发展史上的诸多“第一”:中国最早的内燃机、最早的电机、最早的中文打字机等;新中国
%   第一艘万吨轮、第一艘核潜艇、第一艘气垫船、第一艘水翼艇、自主设计的第一代战斗
%   机、第一枚运载火箭、第一颗人造卫星、第一例心脏二尖瓣分离术、第一例成功移植同种
%   原位肝手术、第一例成功抢救大面积烧伤病人手术等,都凝聚着交大师生和校友的心血智
%   慧。改革开放以来,一批年轻的校友已在世界各地、各行各业崭露头角。}

% {\ifcsname kaishu\endcsname\kaishu\else[无 \cs{kaishu} 字体。]\fi 截至 2011 年 12
%   月 31 日,学校共有 24 个学院 / 直属系(另有继续教育学院、技术学院和国际教育学
%   院),19 个直属单位,12 家附属医院,全日制本科生 16802 人、研究生24495 人(其
%   中博士研究生 5059 人);有专任教师 2979 名,其中教授 835 名;中国科学院院士 15
%   名,中国工程院院士 20 名,中组部“千人计划”49 名,“长江学者”95 名,国家杰出
%   青年基金获得者 80 名,国家重点基础研究发展计划(973 计划)首席科学家 24名,国
%   家重大科学研究计划首席科学家 9名,国家基金委创新研究群体 6 个,教育部创新团队
%   17 个。
  
%   学校现有本科专业 68 个,涵盖经济学、法学、文学、理学、工学、农学、医学、管理学
%   和艺术等九个学科门类;拥有国家级教学及人才培养基地 7 个,国家级校外实践教育基
%   地 5个,国家级实验教学示范中心 5 个,上海市实验教学示范中心 4 个;有国家级教学
%   团队 8个,上海市教学团队 15 个;有国家级教学名师 7 人,上海市教学名师 35 人;
%   有国家级精品课程 46 门,上海市精品课程 117 门;有国家级双语示范课程 7
%   门;2001、2005 和2009 年,作为第一完成单位,共获得国家级教学成果 37 项、上海市
%   教学成果 157项。}
% !TEX root = ../thesis.tex

\chapter{数学与引用文献的标注}

\section{数学}

\subsection{数字和单位}

宏包 \pkg{siunitx} 提供了更好的数字和单位支持:
\begin{itemize}
  \item \num{12345.67890}
  \item \num{1+-2i}
  \item \num{.3e45}
  \item \num{1.654 x 2.34 x 3.430}
  \item \si{kg.m.s^{-1}}
  \item \si{\micro\meter} $\si{\micro\meter}$
  \item \si{\ohm} $\si{\ohm}$
  \item \numlist{10;20}
  \item \numlist{10;20;30}
  \item \SIlist{0.13;0.67;0.80}{\milli\metre}
  \item \numrange{10}{20}
  \item \SIrange{10}{20}{\degreeCelsius}
\end{itemize}

\subsection{数学符号和公式}

微分符号 $\dif$ 应使用正体,本模板提供了 \cs{dif} 命令。除此之外,模板还提供了一
些命令方便使用:
\begin{itemize}
  \item 圆周率 $\uppi$:\verb|\uppi|
  \item 自然对数的底 $\upe$:\verb|\upe|
  \item 虚数单位 $\upi$, $\upj$:\verb|\upi| \verb|\upj|
\end{itemize}

公式应另起一行居中排版。公式后应注明编号,按章顺序编排,编号右端对齐。
\begin{equation}
  \upe^{\upi\uppi} + 1 = 0,
\end{equation}
\begin{equation}
  \frac{\dif^2 u}{\dif t^2} = \int f(x) \dif x.
\end{equation}

公式末尾是需要添加标点符号的,至于用逗号还是句号,取决于公式下面一句是接着公式说的,还是另起一句。
\begin{equation}
		\frac{2h}{\pi}\int_{0}^{\infty}\frac{\sin\left( \omega\delta \right)}{\omega}
		\cos\left( \omega x \right) \dif\omega = 
		\begin{cases}
				h, \ \left| x \right| < \delta, \\
				\frac{h}{2}, \ x = \pm \delta, \\
				0, \ \left| x \right| > \delta.
		\end{cases}
\end{equation}
公式较长时最好在等号“$=$”处转行。
\begin{align}
    & I (X_3; X_4) - I (X_3; X_4 \mid X_1) - I (X_3; X_4 \mid X_2) \nonumber \\
  = & [I (X_3; X_4) - I (X_3; X_4 \mid X_1)] - I (X_3; X_4 \mid \tilde{X}_2) \\
  = & I (X_1; X_3; X_4) - I (X_3; X_4 \mid \tilde{X}_2).
\end{align}

如果在等号处转行难以实现,也可在 $+$、$-$、$\times$、$\div$运算符号处转行,转行
时运算符号仅书写于转行式前,不重复书写。
\begin{multline}
  \frac{1}{2} \Delta (f_{ij} f^{ij}) =
    2 \left(\sum_{i<j} \chi_{ij}(\sigma_{i} - \sigma_{j})^{2}
    + f^{ij} \nabla_{j} \nabla_{i} (\Delta f) \right. \\
  \left. + \nabla_{k} f_{ij} \nabla^{k} f^{ij} +
    f^{ij} f^{k} \left[2\nabla_{i}R_{jk}
    - \nabla_{k} R_{ij} \right] \vphantom{\sum_{i<j}} \right).
\end{multline}

\subsection{定理环境}

示例文件中使用 \pkg{ntheorem} 宏包配置了定理、引理和证明等环境。用户也可以使用
\pkg{amsthm} 宏包。

这里举一个“定理”和“证明”的例子。
\begin{theorem}[留数定理]
\label{thm:res}
  假设 $U$ 是复平面上的一个单连通开子集,$a_1, \ldots, a_n$ 是复平面上有限个点,
  $f$ 是定义在 $U \backslash \{a_1, \ldots, a_n\}$ 上的全纯函数,如果 $\gamma$
  是一条把 $a_1, \ldots, a_n$ 包围起来的可求长曲线,但不经过任何一个 $a_k$,并且
  其起点与终点重合,那么:

  \begin{equation}
    \label{eq:res}
    \ointop_\gamma f(z)\, \dif z = 2\uppi \upi \sum_{k=1}^n \operatorname{I}(\gamma, a_k) \operatorname{Res}(f, a_k).
  \end{equation}

  如果 $\gamma$ 是若尔当曲线,那么 $\operatorname{I}(\gamma, a_k) = 1$,因此:

  \begin{equation}
    \label{eq:resthm}
    \ointop_\gamma f(z)\, \dif z = 2\uppi \upi \sum_{k=1}^n \operatorname{Res}(f, a_k).
  \end{equation}

  在这里,$\operatorname{Res}(f, a_k)$ 表示 $f$ 在点 $a_k$ 的留数,
  $\operatorname{I}(\gamma, a_k)$ 表示 $\gamma$ 关于点 $a_k$ 的卷绕数。卷绕数是
  一个整数,它描述了曲线 $\gamma$ 绕过点 $a_k$ 的次数。如果 $\gamma$ 依逆时针方
  向绕着 $a_k$ 移动,卷绕数就是一个正数,如果 $\gamma$ 根本不绕过 $a_k$,卷绕数
  就是零。

  定理~\ref{thm:res} 的证明。

  \begin{proof}
    首先,由……

    其次,……

    所以……
  \end{proof}
\end{theorem}

\section{引用文献的标注}

按照教务处的要求,参考文献外观应符合国标 GB/T 7714 的要求。模版使用 \BibLaTeX\
配合 \pkg{biblatex-gb7714-2015} 样式包
\footnote{\url{https://www.ctan.org/pkg/biblatex-gb7714-2015}}
控制参考文献的输出样式,后端采用 \pkg{biber} 管理文献。

请注意 \pkg{biblatex-gb7714-2015} 宏包 2016 年 9 月才加入 CTAN,如果你使用的
\TeX\ 系统版本较旧,可能没有包含 \pkg{biblatex-gb7714-2015} 宏包,需要手动安装。
\BibLaTeX\ 与 \pkg{biblatex-gb7714-2015} 目前在活跃地更新,为避免一些兼容性问
题,推荐使用较新的版本。

正文中引用参考文献时,使用 \verb|\cite{key1,key2,key3...}| 可以产生“上标引用的
参考文献”,如 \cite{Meta_CN,chen2007act,DPMG}。使用
\verb|\parencite{key1,key2,key3...}| 则可以产生水平引用的参考文献,例如
\parencite{JohnD,zhubajie,IEEE-1363}。请看下面的例子,将会穿插使用水平的和上标的
参考文献:关于书的\parencite{Meta_CN,JohnD,IEEE-1363},关于期刊的
\cite{chen2007act,chen2007ewi},会议论文 \parencite{DPMG,kocher99,cnproceed},硕
士学位论文\parencite{zhubajie,metamori2004},博士学位论文
\cite{shaheshang,FistSystem01,bai2008},标准文件 \parencite{IEEE-1363},技术报告
\cite{NPB2},电子文献 \parencite{xiaoyu2001, CHRISTINE1998},用户手册
\parencite{RManual}。

当需要将参考文献条目加入到文献表中但又不在正文中引用,可以使用
\verb|\nocite{key1,key2,key3...}|。使用 \verb|\nocite{*}| 可以将参考文献数据库中
的所有条目加入到文献表中。

% !TeX root = ../main.tex

\chapter{浮动体}

\section{插图}

插图功能是利用 \TeX{} 的特定编译程序提供的机制实现的,不同的编译程序支持不同的图
形方式。有的同学可能听说“\LaTeX{} 只支持 EPS”,事实上这种说法是不准确的。\XeTeX{}
可以很方便地插入 EPS、PDF、PNG、JPEG 格式的图片。

一般图形都是处在浮动环境中。之所以称为浮动是指最终排版效果图形的位置不一定与源文
件中的位置对应,这也是刚使用 \LaTeX{} 同学可能遇到的问题。如果要强制固定浮动图形
的位置,请使用 \pkg{float} 宏包,它提供了 \texttt{[H]} 参数。

\subsection{单个图形}

图要有图题,研究生图题采用中英文对照,并置于图的编号之后,图的编号和图题应置于图
下方的居中位置。引用图应在图题右上角标出文献来源。文中必须有关于本插图的提示,如
“见图~\ref{fig:energy-distrib}”、“如图~\ref{fig:energy-distrib} 所示”等。该页空
白不够排写该图整体时,则可将其后文字部分提前排写,将图移到次页。

\begin{figure}[!htp]
  \centering
  \begin{tikzpicture}
    \begin{axis}[
      width=12cm,
      height=9cm,
      xmin=0, xmax=7,
      xlabel={$r$ (\unit{\milli\metre})},
      ymin=-1000, ymax=11000,
      ylabel={Energy (\unit[per-mode=symbol]{\watt\per\cubic\metre})},
      scaled ticks=false,
      tick label style={
        /pgf/number format/1000 sep=,
        font={\zihao{-5}},
      },
      minor tick num=1,
      tick pos=left,
      tick align=outside,
      tick style={thin,black},
    ]
      \addplot [only marks,mark=square*] 
        table [x={radial}, y={energy}, col sep=comma] 
        {./assets/energy-distrib.csv};
      \node at (2,6000) 
        {$q_{v}=\dfrac{\sigma\omega^{2}|\mathbf{A}|^{2}}{2}$};
    \end{axis}
  \end{tikzpicture}
  \bicaption{内热源沿径向的分布}{Energy distribution along radial}
  \label{fig:energy-distrib}
\end{figure}

\subsection{多个图形}

简单插入多个图形的例子如图~\ref{fig:SRR} 所示。这两个水平并列放置的子图共用一个
图形计数器,没有各自的子图题。

\begin{figure}[!htp]
  \centering
  \includegraphics[height=2.5cm]{example-image-a.pdf}%
  \hfil\hfil%
  \includegraphics[height=2.5cm]{example-image-b.pdf}
  \bicaption{中文题图}{English caption}
  \label{fig:SRR}
\end{figure}

如果多个图形相互独立,并不共用一个图形计数器,那么用 \texttt{minipage} 或者
\texttt{parbox} 就可以,如图~\ref{fig:parallel1} 与图~\ref{fig:parallel2}。

\begin{figure}[!htp]
  \centering
  \begin{minipage}{0.5\textwidth}
    \centering
    \includegraphics[height=2.5cm]{example-image-a.pdf}
    \caption{并排第一个图}
    \label{fig:parallel1}
  \end{minipage}%
  \begin{minipage}{0.5\textwidth}
    \centering
    \includegraphics[height=2.5cm]{example-image-b.pdf}
    \caption{并排第二个图}
    \label{fig:parallel2}
  \end{minipage}
\end{figure}

如果要为共用一个计数器的多个子图添加子图题,建议使用较新的 \pkg{subcaption} 宏
包,不建议使用 \pkg{subfigure} 或 \pkg{subfig} 等宏包。

推荐使用 \pkg{subcaption} 宏包的 \cs{subcaptionbox} 并排子图,子图题置于子图之
下,子图号用 a)、b) 等表示。也可以使用 \pkg{subcaption} 宏包的 \cs{subcaption}
(放在 minipage中,用法同 \cs{caption})。

\pkg{subcaption} 宏包也提供了 \pkg{subfigure} 和 \pkg{subtable} 环境,如
图~\ref{fig:subfigure}。

\begin{figure}[!htp]
  \centering
  \begin{subfigure}{0.45\textwidth}
    \centering
    \includegraphics[height=2.5cm]{example-image-a.pdf}
    \caption{子图甲}
  \end{subfigure}%
  \hfil\hfil%
  \begin{subfigure}{0.45\textwidth}
    \centering
    \includegraphics[height=2cm]{example-image-b.pdf}
    \caption{子图乙。注意这个图略矮些,subfigure 中同一行的子图在顶端对齐。}
  \end{subfigure}
  \caption{包含子图题的范例(使用 subfigure)}
  \label{fig:subfigure}
\end{figure}

搭配 \pkg{bicaption} 宏包时,可以启用 \cs{subcaptionbox} 和 \cs{subcaption} 的双
语变种 \cs{bisubcaptionbox} 和 \cs{bisubcaption},如图~\ref{fig:bisubcaptionbox}
所示。

\begin{figure}[!hbtp]
  \centering
  \bisubcaptionbox{$R_3 = 1.5\text{mm}$ 时轴承的压力分布云图}%
                  {Pressure contour of bearing when $R_3 = 1.5\text{mm}$}%
                  [0.4\textwidth]{\includegraphics[height=2.5cm]{example-image-a.pdf}}%
  \hfil\hfil%
  \bisubcaptionbox{$R_3 = 2.5\text{mm}$ 时轴承的压力分布云图}%
                  {Pressure contour of bearing when $R_3 = 2.5\text{mm}$}%
                  [0.4\textwidth]{\includegraphics[height=2.5cm]{example-image-b.pdf}}
  \bicaption{包含子图题的范例(使用 subcaptionbox)}
            {Example with subcaptionbox}
  \label{fig:bisubcaptionbox}
\end{figure}


\section{表格}

\subsection{基本表格}

编排表格应简单明了,表达一致,明晰易懂,表文呼应、内容一致。表题置于表上,研究生
学位论文可以用中、英文两种文字居中排写,中文在上,也可以只用中文。

表格的编排建议采用国际通行的三线表\footnote{三线表,以其形式简洁、功能分明、阅读
方便而在科技论文中被推荐使用。三线表通常只有 3 条线,即顶线、底线和栏目线,没有
竖线。}。三线表可以使用 \pkg{booktabs} 提供的 \cs{toprule}、\cs{midrule} 和
\cs{bottomrule}。它们与 \pkg{longtable} 能很好的配合使用。

\begin{table}[!hpt]
  \caption[一个颇为标准的三线表]{一个颇为标准的三线表\footnotemark}
  \label{tab:firstone}
  \centering
  \begin{tabular}{@{}llr@{}} \toprule
    \multicolumn{2}{c}{Item} \\ \cmidrule(r){1-2}
    Animal & Description & Price (\$)\\ \midrule
    Gnat  & per gram  & 13.65 \\
          & each      & 0.01 \\
    Gnu   & stuffed   & 92.50 \\
    Emu   & stuffed   & 33.33 \\
    Armadillo & frozen & 8.99 \\ \bottomrule
  \end{tabular}
\end{table}
\footnotetext{这个例子来自
  \href{https://mirrors.sjtug.sjtu.edu.cn/ctan/macros/latex/contrib/booktabs/booktabs.pdf}%
  {《Publication quality tables in LaTeX》}(\pkg{booktabs} 宏包的文档)。这也是
  一个在表格中使用脚注的例子,请留意与 \pkg{threeparttable} 实现的效果有何不
  同。}

\subsection{复杂表格}

我们经常会在表格下方标注数据来源,或者对表格里面的条目进行解释。可以用
\pkg{threeparttable} 实现带有脚注的表格,如表~\ref{tab:footnote}。

\begin{table}[!htpb]
  \bicaption{一个带有脚注的表格的例子}{A Table with footnotes}
  \label{tab:footnote}
  \centering
  \begin{threeparttable}[b]
     \begin{tabular}{ccd{4}cccc}
      \toprule
      \multirow{2}*{total} & \multicolumn{2}{c}{20\tnote{a}} & \multicolumn{2}{c}{40} & \multicolumn{2}{c}{60} \\
      \cmidrule(lr){2-3}\cmidrule(lr){4-5}\cmidrule(lr){6-7}
      & www & \multicolumn{1}{c}{k} & www & k & www & k \\ % 使用说明符 d 的列会自动进入数学模式,使用 \multicolumn 对文字表头做特殊处理
      \midrule
      & $\underset{(2.12)}{4.22}$ & 120.0140\tnote{b} & 333.15 & 0.0411 & 444.99 & 0.1387 \\
      & 168.6123 & 10.86 & 255.37 & 0.0353 & 376.14 & 0.1058 \\
      & 6.761    & 0.007 & 235.37 & 0.0267 & 348.66 & 0.1010 \\
      \bottomrule
    \end{tabular}
    \begin{tablenotes}
    \item [a] the first note.
    \item [b] the second note.
    \end{tablenotes}
  \end{threeparttable}
\end{table}

如某个表需要转页接排,可以用 \pkg{longtable} 实现。接排时表题省略,表头应重复书
写,并在右上方写“续表 xx”,如表~\ref{tab:performance}。

\begin{ThreePartTable}
  \begin{TableNotes}
    \item[a] 一个脚注
    \item[b] 另一个脚注
  \end{TableNotes}
  \begin{longtable}[c]{c*{6}{r}}
    \bicaption{实验数据}{Experimental data}
    \label{tab:performance} \\
    \toprule
    测试程序 & \multicolumn{1}{c}{正常运行} & \multicolumn{1}{c}{同步}
      & \multicolumn{1}{c}{检查点} & \multicolumn{1}{c}{卷回恢复}
      & \multicolumn{1}{c}{进程迁移} & \multicolumn{1}{c}{检查点} \\
    & \multicolumn{1}{c}{时间 (s)} & \multicolumn{1}{c}{时间 (s)}
      & \multicolumn{1}{c}{时间 (s)} & \multicolumn{1}{c}{时间 (s)}
      & \multicolumn{1}{c}{时间 (s)} &  文件(KB)\\
    \midrule
    \endfirsthead
    \multicolumn{7}{l}{\textbf{续表~\thetable}} \\
    % 英语论文:\multicolumn{7}{r}{\textbf{Table~\thetable~(continued)}} \\
    \toprule
    测试程序 & \multicolumn{1}{c}{正常运行} & \multicolumn{1}{c}{同步}
      & \multicolumn{1}{c}{检查点} & \multicolumn{1}{c}{卷回恢复}
      & \multicolumn{1}{c}{进程迁移} & \multicolumn{1}{c}{检查点} \\
    & \multicolumn{1}{c}{时间 (s)} & \multicolumn{1}{c}{时间 (s)}
      & \multicolumn{1}{c}{时间 (s)} & \multicolumn{1}{c}{时间 (s)}
      & \multicolumn{1}{c}{时间 (s)}&  文件(KB)\\
    \midrule
    \endhead
    \hline
    \multicolumn{7}{r}{续下页}
    \endfoot
    \insertTableNotes
    \endlastfoot
    CG.A.2 & 23.05 & 0.002 & 0.116 & 0.035 & 0.589 & 32491 \\
    CG.A.4 & 15.06 & 0.003 & 0.067 & 0.021 & 0.351 & 18211 \\
    CG.A.8 & 13.38 & 0.004 & 0.072 & 0.023 & 0.210 & 9890 \\
    CG.B.2 & 867.45 & 0.002 & 0.864 & 0.232 & 3.256 & 228562 \\
    CG.B.4 & 501.61 & 0.003 & 0.438 & 0.136 & 2.075 & 123862 \\
    CG.B.8 & 384.65 & 0.004 & 0.457 & 0.108 & 1.235 & 63777 \\
    MG.A.2 & 112.27 & 0.002 & 0.846 & 0.237 & 3.930 & 236473 \\
    MG.A.4 & 59.84 & 0.003 & 0.442 & 0.128 & 2.070 & 123875 \\
    MG.A.8 & 31.38 & 0.003 & 0.476 & 0.114 & 1.041 & 60627 \\
    MG.B.2 & 526.28 & 0.002 & 0.821 & 0.238 & 4.176 & 236635 \\
    MG.B.4 & 280.11 & 0.003 & 0.432 & 0.130 & 1.706 & 123793 \\
    MG.B.8 & 148.29 & 0.003 & 0.442 & 0.116 & 0.893 & 60600 \\
    LU.A.2 & 2116.54 & 0.002 & 0.110 & 0.030 & 0.532 & 28754 \\
    LU.A.4 & 1102.50 & 0.002 & 0.069 & 0.017 & 0.255 & 14915 \\
    LU.A.8 & 574.47 & 0.003 & 0.067 & 0.016 & 0.192 & 8655 \\
    LU.B.2 & 9712.87 & 0.002 & 0.357 & 0.104 & 1.734 & 101975 \\
    LU.B.4 & 4757.80 & 0.003 & 0.190 & 0.056 & 0.808 & 53522 \\
    LU.B.8 & 2444.05 & 0.004 & 0.222 & 0.057 & 0.548 & 30134 \\
    EP.A.2 & 123.81 & 0.002 & 0.010 & 0.003 & 0.074 & 1834 \\
    EP.A.4 & 61.92 & 0.003 & 0.011 & 0.004 & 0.073 & 1743 \\
    EP.A.8 & 31.06 & 0.004 & 0.017 & 0.005 & 0.073 & 1661 \\
    EP.B.2 & 495.49 & 0.001 & 0.009 & 0.003 & 0.196 & 2011 \\
    EP.B.4 & 247.69 & 0.002 & 0.012 & 0.004 & 0.122 & 1663 \\
    EP.B.8 & 126.74 & 0.003 & 0.017 & 0.005 & 0.083 & 1656 \\
    SP.A.2 & 123.81 & 0.002 & 0.010 & 0.003 & 0.074 & 1854 \\
    SP.A.4 & 51.92 & 0.003 & 0.011 & 0.004 & 0.073 & 1543 \\
    SP.A.8 & 31.06 & 0.004 & 0.017 & 0.005 & 0.073 & 1671 \\
    SP.B.2 & 495.49 & 0.001 & 0.009 & 0.003 & 0.196 & 2411 \\
    SP.B.4 \tnote{a} & 247.69 & 0.002 & 0.014 & 0.006 & 0.152 & 2653 \\
    SP.B.8 \tnote{b} & 126.74 & 0.003 & 0.017 & 0.005 & 0.082 & 1755 \\
    \bottomrule
  \end{longtable}
\end{ThreePartTable}

\section{算法环境}

算法环境可以使用 \pkg{algorithms} 宏包或者较新的 \pkg{algorithm2e} 实现。
算法~\ref{algo:algorithm} 是一个使用 \pkg{algorithm2e} 的例子。关于排版算法环境
的具体方法,请阅读相关宏包的官方文档。

\begin{algorithm}[htb]
  \caption{算法示例}
  \label{algo:algorithm}
  \SetAlgoLined
  \KwData{this text}
  \KwResult{how to write algorithm with \LaTeXe }

  initialization\;
  \While{not at end of this document}{
    read current\;
    \eIf{understand}{
      go to next section\;
      current section becomes this one\;
    }{
      go back to the beginning of current section\;
    }
  }
\end{algorithm}

\section{代码环境}

我们可以在论文中插入算法,但是不建议插入大段的代码。如果确实需要插入代码,建议使
用 \pkg{listings} 宏包。

\begin{codeblock}[language=C]
#include <stdio.h>
#include <unistd.h>
#include <sys/types.h>
#include <sys/wait.h>

int main() {
  pid_t pid;

  switch ((pid = fork())) {
  case -1:
    printf("fork failed\n");
    break;
  case 0:
    /* child calls exec */
    execl("/bin/ls", "ls", "-l", (char*)0);
    printf("execl failed\n");
    break;
  default:
    /* parent uses wait to suspend execution until child finishes */
    wait((int*)0);
    printf("is completed\n");
    break;
  }

  return 0;
}
\end{codeblock}

%# -*- coding: utf-8-unix -*-
%%==================================================
%% conclusion.tex for SJTUThesis
%% Encoding: UTF-8
%%==================================================

\begin{summary}

这里是全文总结内容。

2015年2月28日,中央在北京召开全国精神文明建设工作表彰暨学雷锋志愿服务大会,公布全国文明城市(区)、文明村镇、文明单位名单。上海交通大学荣获全国文明单位称号。         

全国文明单位这一荣誉是对交大人始终高度重视文明文化工作的肯定,是对交大长期以来文明创建工作成绩的褒奖。在学校党委、文明委的领导下,交大坚持将文明创建工作纳入学校建设世界一流大学的工作中,全体师生医护员工群策群力、积极开拓,落实国家和上海市有关文明创建的各项要求,以改革创新、科学发展为主线,以质量提升为目标,聚焦文明创建工作出现的重点和难点,优化文明创建工作机制,传播学校良好形象,提升社会美誉度,显著增强学校软实力。2007至2012年间,上海交大连续三届荣获“上海市文明单位”称号,成为创建全国文明单位的新起点。         

上海交大自启动争创全国文明单位工作以来,凝魂聚气、改革创新,积极培育和践行社会主义核心价值观。坚持统筹兼顾、多措并举,将争创全国文明单位与学校各项中心工作紧密结合,着力构建学校文明创建新格局,不断提升师生医护员工文明素养,以“冲击世界一流大学汇聚强大精神动力”为指导思想,以“聚焦改革、多元推进、以评促建、丰富内涵、彰显特色”为工作原则,并由全体校领导群策领衔“党的建设深化、思想教育深入、办学成绩显著、大学文化丰富、校园环境优化、社会责任担当”六大板块共28项重点突破工作,全面展现近年来交大文明创建工作的全貌和成就。         

进入新阶段,学校将继续开拓文明创建工作新格局,不断深化工作理念和工作实践,创新工作载体、丰富活动内涵、凸显创建成效,积极服务于学校各项中心工作和改革发展的大局面,在上级党委、文明委的关心下,在学校党委的直接领导下,与时俱进、开拓创新,为深化内涵建设、加快建成世界一流大学、推动国家进步和社会发展而努力奋斗!       

上海交通大学医学院附属仁济医院也获得全国文明单位称号。      

\end{summary}


%TC:ignore

% 参考文献
\printbibliography[heading=bibintoc]

% 附录
\appendix

% 附录中图表不加入索引
\captionsetup{list=no}

% 附录内容
% !TEX root = ../thesis.tex

\chapter{Maxwell Equations}

选择二维情况,有如下的偏振矢量:
\begin{subequations}
  \begin{align}
    {\bf E} &= E_z(r, \theta) \hat{\bf z}, \\
    {\bf H} &= H_r(r, \theta) \hat{\bf r} + H_\theta(r, \theta) \hat{\bm\theta}.
  \end{align}
\end{subequations}
对上式求旋度:
\begin{subequations}
  \begin{align}
    \nabla \times {\bf E} &= \frac{1}{r} \frac{\partial E_z}{\partial\theta}
      \hat{\bf r} - \frac{\partial E_z}{\partial r} \hat{\bm\theta}, \\
    \nabla \times {\bf H} &= \left[\frac{1}{r} \frac{\partial}{\partial r}
      (r H_\theta) - \frac{1}{r} \frac{\partial H_r}{\partial\theta} \right]
      \hat{\bf z}.
  \end{align}
\end{subequations}
因为在柱坐标系下,$\overline{\overline\mu}$ 是对角的,所以 Maxwell 方程组中电场
$\bf E$ 的旋度:
\begin{subequations}
  \begin{align}
    & \nabla \times {\bf E} = \upi \omega {\bf B}, \\
    & \frac{1}{r} \frac{\partial E_z}{\partial\theta} \hat{\bf r} -
      \frac{\partial E_z}{\partial r}\hat{\bm\theta} = \upi \omega \mu_r H_r
      \hat{\bf r} + \upi \omega \mu_\theta H_\theta \hat{\bm\theta}.
  \end{align}
\end{subequations}
所以 $\bf H$ 的各个分量可以写为:
\begin{subequations}
  \begin{align}
    H_r &= \frac{1}{\upi \omega \mu_r} \frac{1}{r}
      \frac{\partial E_z}{\partial\theta}, \\
    H_\theta &= -\frac{1}{\upi \omega \mu_\theta}
      \frac{\partial E_z}{\partial r}.
  \end{align}
\end{subequations}
同样地,在柱坐标系下,$\overline{\overline\epsilon}$ 是对角的,所以 Maxwell 方程
组中磁场 $\bf H$ 的旋度:
\begin{subequations}
  \begin{align}
    & \nabla \times {\bf H} = -\upi \omega {\bf D}, \\
    & \left[\frac{1}{r} \frac{\partial}{\partial r}(r H_\theta) - \frac{1}{r}
      \frac{\partial H_r}{\partial\theta} \right] \hat{\bf z} = -\upi \omega
      {\overline{\overline\epsilon}} {\bf E} = -\upi \omega \epsilon_z E_z
      \hat{\bf z}, \\
    & \frac{1}{r} \frac{\partial}{\partial r}(r H_\theta) - \frac{1}{r}
      \frac{\partial H_r}{\partial\theta} = -\upi \omega \epsilon_z E_z.
  \end{align}
\end{subequations}
由此我们可以得到关于 $E_z$ 的波函数方程:
\begin{equation}
  \frac{1}{\mu_\theta \epsilon_z} \frac{1}{r} \frac{\partial}{\partial r}
  \left(r \frac{\partial E_z}{\partial r} \right) + \frac{1}{\mu_r \epsilon_z}
  \frac{1}{r^2} \frac{\partial^2E_z}{\partial\theta^2} +\omega^2 E_z = 0.
\end{equation}

% !TEX root = ../main.tex

\chapter{绘制流程图}

图~\ref{fig:flow_chart} 是一张流程图示意。使用 \pkg{tikz} 环境,搭配四种预定义节
点(\verb|startstop|、\verb|process|、\verb|decision| 和 \verb|io|),可以容易地
绘制出流程图。

\begin{figure}[!htp]
  \centering
  
% 定义流程图节点
\tikzstyle{startstop} = [
  rectangle,
  rounded corners,
  minimum width=4em,
  text centered,
  inner sep=1.5ex,
  draw
]
\tikzstyle{io} = [
  trapezium,
  trapezium left angle=75,
  trapezium right angle=105,
  minimum width=4em,
  text centered,
  inner sep=1.5ex,
  draw
]
\tikzstyle{process} = [
  rectangle,
  minimum width=4em,
  text centered,
  inner sep=1.5ex,
  draw
]
\tikzstyle{decision} = [
  diamond,
  minimum width=4em,
  aspect=2,
  text centered,
  draw
]
\tikzstyle{arrow} = [-{LaTeX}]

\begin{tikzpicture}[node distance=1.5cm, every node/.style={font=\footnotesize}]
  % 设置节点
  \node[startstop] (pic) {待测图片};
  \node[io, below of=pic] (bg) {读取背景};
  \node[process, below of=bg] (pair) {匹配特征点对};
  \node[decision, below of=pair, yshift=-2ex] (threshold) {多于阈值};
  \node[decision, right of=threshold, xshift=3cm] (clear) {清晰?};
  \node[process, right of=pair, xshift=3cm] (capture) {重采};
  \node[process, below of=threshold, yshift=-2ex] (matrix_p) {透视变换矩阵};
  \node[process, right of=matrix_p, xshift=3cm] (matrix_a) {仿射变换矩阵};
  \node[process, below of=matrix_p] (reg) {图像修正};
  \node[startstop, below of=reg] (return) {配准结果};
    
  % 连接节点
  \draw[arrow] (pic) -- (bg);
  \draw[arrow] (bg) -- (pair);
  \draw[arrow] (pair) -- (threshold);

  \draw[arrow] (threshold) -- node[above] {否} (clear);

  \draw[arrow] (clear) -- node[right] {否} (capture);
  \draw[arrow] (capture) |- (pic);
  \draw[arrow] (clear) -- node[right] {是} (matrix_a);
  \draw[arrow] (matrix_a) |- (reg);

  \draw[arrow] (threshold) -- node[left] {是} (matrix_p);
  \draw[arrow] (matrix_p) -- (reg);
  \draw[arrow] (reg) -- (return);
\end{tikzpicture}

  \bicaption{绘制流程图效果}{Flow chart}
  \label{fig:flow_chart}
\end{figure}


% 结尾部分
\backmatter

% 用于盲审的论文需隐去致谢、发表论文、科研成果、简历

% 致谢
% !TEX root = ../main.tex

\begin{acknowledgements}
  感谢那位最先制作出博士学位论文 \LaTeX{} 模板的物理系同学!

  感谢 William Wang 同学对模板移植做出的贡献!

  感谢 \href{https://github.com/weijianwen}{@weijianwen} 学长开创性的工作!

  感谢 \href{https://github.com/sjtug}{@sjtug} 对 0.10 及之后版本的开发和维护工作!

  感谢所有为模板贡献过代码的\href{https://github.com/sjtug/SJTUThesis/graphs/contributors}{同学们}, 以及所有测试和使用模板的各位同学!

  感谢 \LaTeX 和 \href{https://github.com/sjtug/SJTUThesis}{SJTUThesis},帮我节省了不少时间。
\end{acknowledgements}


% 发表论文及科研成果
% 盲审论文中,发表论文及科研成果等仅以第几作者注明即可,不要出现作者或他人姓名
% !TEX root = ../main.tex

%TC:ignore

\begin{achievements}
  \item 第一发明人,“永动机”,专利申请号202510149890.0
\end{achievements}

\begin{achievements*}
  \item 第一发明人,“永动机”,专利申请号XXXXXXXXXXXX.X
\end{achievements*}

%TC:endignore


% 简历
%%==================================================
%% resume.tex for SJTU Bachelor Thesis
%% version: 0.5.2
%% Encoding: UTF-8
%%==================================================

\begin{resume}

\begin{resumesection}{基本情况}
xxx,男,上海人,19XX 年~XX 月出生,未婚,
上海交通大学XX系在读。
\end{resumesection}

\begin{resumelist}{教育状况}
XXXX 年~9 月至~XXXX 年~7 月,上海交通大学, 本科,专业:XXXX

XXXX 年~9 月至~XXXX 年~7 月,上海交通大学, 硕士研究生,专业:XXXX

XXXX 年~9 月至~XXXX 年~7 月,上海交通大学,
博士研究生(提前攻读博士),专业:XXXX
\end{resumelist}

\begin{resumelist}{工作经历}
无。
\end{resumelist}

\begin{resumelist}{研究兴趣}
XXXXXXX。
\end{resumelist}

\begin{resumelist}{联系方式}
通讯地址:上海市闵行区东川路800号,上海交通大学

邮编:200240

E-mail: abcde@sjtu.edu.cn
\end{resumelist}

\end{resume}


% 学士学位论文要求在最后有一个大摘要,单独编页码
\input{contents/digest}

%TC:endignore

\end{document}
