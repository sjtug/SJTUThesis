% \iffalse meta-comment
%
% Copyright (C) 2018--2021 by Alexara Wu <alexarawu@outlook.com>
%
% This work may be distributed and/or modified under the
% conditions of the LaTeX Project Public License, either
% version 1.3c of this license or (at your option) any later
% version. The latest version of this license is in:
%
%   http://www.latex-project.org/lppl.txt
%
% and version 1.3 or later is part of all distributions of
% LaTeX version 2005/12/01 or later.
%
% This work has the LPPL maintenance status `maintained'.
%
% The Current Maintainer of this work is Alexara Wu.
%
%<*internal>
\iffalse
%</internal>
%
%<*internal>
\fi
\begingroup
  \def\NameOfLaTeXe{LaTeX2e}
\expandafter\endgroup\ifx\NameOfLaTeXe\fmtname\else
\csname fi\endcsname
%</internal>
%
%<*install>
\input l3docstrip.tex
\keepsilent
\askforoverwritefalse

\preamble

    Copyright (C) 2018--2021 by Alexara Wu <alexarawu@outlook.com>

    This work may be distributed and/or modified under the
    conditions of the LaTeX Project Public License, either
    version 1.3c of this license or (at your option) any later
    version. The latest version of this license is in:

      http://www.latex-project.org/lppl.txt

    and version 1.3 or later is part of all distributions of
    LaTeX version 2005/12/01 or later.

    This work has the LPPL maintenance status `maintained'.

    The Current Maintainer of this work is Alexara Wu.

\endpreamble

\generate{
  \usedir{tex/latex/sjtuthesis}
    \file{\jobname.cls}                 {\from{\jobname.dtx}{entry}}
    \file{\jobname-postgraduate-zh.cls} {\from{\jobname.dtx}{class,pg,zh}}
    \file{\jobname-postgraduate-en.cls} {\from{\jobname.dtx}{class,pg,en}}
    \file{\jobname-undergraduate-zh.cls}{\from{\jobname.dtx}{class,ug,zh}}
    \file{\jobname-undergraduate-en.cls}{\from{\jobname.dtx}{class,ug,en}}
    \file{\jobname-name.cfg}            {\from{\jobname.dtx}{name}}
    \file{sjtu-text-font-termes.def}    {\from{\jobname.dtx}{textfont,termes}}
    \file{sjtu-text-font-pagella.def}   {\from{\jobname.dtx}{textfont,pagella}}
    \file{sjtu-text-font-lm.def}        {\from{\jobname.dtx}{textfont,lm}}
    \file{sjtu-text-font-stixtwo.def}   {\from{\jobname.dtx}{textfont,stixtwo}}
    \file{sjtu-text-font-xits.def}      {\from{\jobname.dtx}{textfont,xits}}
    \file{sjtu-text-font-newcm.def}     {\from{\jobname.dtx}{textfont,newcm}}
    \file{sjtu-text-font-cambria.def}   {\from{\jobname.dtx}{textfont,cambria}}
    \file{sjtu-text-font-times.def}     {\from{\jobname.dtx}{textfont,times}}
    \file{sjtu-math-font-termes.def}    {\from{\jobname.dtx}{mathfont,termes}}
    \file{sjtu-math-font-pagella.def}   {\from{\jobname.dtx}{mathfont,pagella}}
    \file{sjtu-math-font-lm.def}        {\from{\jobname.dtx}{mathfont,lm}}
    \file{sjtu-math-font-stixtwo.def}   {\from{\jobname.dtx}{mathfont,stixtwo}}
    \file{sjtu-math-font-xits.def}      {\from{\jobname.dtx}{mathfont,xits}}
    \file{sjtu-math-font-newcm.def}     {\from{\jobname.dtx}{mathfont,newcm}}
    \file{sjtu-math-font-cambria.def}   {\from{\jobname.dtx}{mathfont,cambria}}
    \file{sjtu-cjk-font-windows.def}    {\from{\jobname.dtx}{cjkfont,windows}}
    \file{sjtu-cjk-font-mac.def}        {\from{\jobname.dtx}{cjkfont,mac}}
    \file{sjtu-cjk-font-ubuntu.def}     {\from{\jobname.dtx}{cjkfont,ubuntu}}
    \file{sjtu-cjk-font-adobe.def}      {\from{\jobname.dtx}{cjkfont,adobe}}
    \file{sjtu-cjk-font-fandol.def}     {\from{\jobname.dtx}{cjkfont,fandol}}
    \file{sjtu-cjk-font-founder.def}    {\from{\jobname.dtx}{cjkfont,founder}}
%</install>
%<*internal>
  \usedir{source/latex/sjtuthesis}
    \file{\jobname.ins}                 {\from{\jobname.dtx}{install}}
%</internal>
%<*install>
}

\obeyspaces
\Msg{*************************************************************}
\Msg{*                                                           *}
\Msg{* To finish the installation you have to move the following *}
\Msg{* files into a directory searched by TeX:                   *}
\Msg{*                                                           *}
\Msg{* The recommended directory is TDS:tex/latex/sjtuthesis     *}
\Msg{*                                                           *}
\Msg{*     sjtuthesis.cls                                        *}
\Msg{*                                                           *}
\Msg{* To produce the documentation, run the file sjtuthesis.dtx *}
\Msg{* through XeLaTeX.                                          *}
\Msg{*                                                           *}
\Msg{* Happy TeXing!                                             *}
\Msg{*                                                           *}
\Msg{*************************************************************}

\endbatchfile
%</install>
%
%<*internal>
\fi
%</internal>
%
%<entry|class>\NeedsTeXFormat{LaTeX2e}
%<entry|class>\RequirePackage{expl3}
%<*!(driver|install)>
%<+!driver>\GetIdInfo$Id$
%<entry>  {Thesis template for Shanghai Jiao Tong University}
%<entry>\ProvidesExplClass{sjtuthesis}
%<pg&zh>  {Thesis template for Shanghai Jiao Tong University (for postgraduates)}
%<pg&zh>\ProvidesExplClass{sjtuthesis-postgraduate-zh}
%<pg&en>  {Thesis template for Shanghai Jiao Tong University (for postgraduates)}
%<pg&en>\ProvidesExplClass{sjtuthesis-postgraduate-en}
%<ug&zh>  {Thesis template for Shanghai Jiao Tong University (for undergraduates)}
%<ug&zh>\ProvidesExplClass{sjtuthesis-undergraduate-zh}
%<ug&en>  {Thesis template for Shanghai Jiao Tong University (for undergraduates)}
%<ug&en>\ProvidesExplClass{sjtuthesis-undergraduate-en}
%<name>  {Names definition (SJTUThesis)}
%<name>\ProvidesExplFile{sjtuthesis-name.cfg}
%<textfont&termes>  {Termes text fonts definition (SJTUThesis)}
%<textfont&termes>\ProvidesExplFile{sjtu-text-font-termes.def}
%<textfont&pagella>  {Pagella text fonts definition (SJTUThesis)}
%<textfont&pagella>\ProvidesExplFile{sjtu-text-font-pagella.def}
%<textfont&stixtwo>  {STIX Two text fonts definition (SJTUThesis)}
%<textfont&stixtwo>\ProvidesExplFile{sjtu-text-font-stixtwo.def}
%<textfont&xits>  {XITS text fonts definition (SJTUThesis)}
%<textfont&xits>\ProvidesExplFile{sjtu-text-font-xits.def}
%<textfont&lm>  {Latin Modern text fonts definition (SJTUThesis)}
%<textfont&lm>\ProvidesExplFile{sjtu-text-font-lm.def}
%<textfont&newcm>  {New Computer Modern text fonts definition (SJTUThesis)}
%<textfont&newcm>\ProvidesExplFile{sjtu-text-font-newcm.def}
%<textfont&cambria>  {Cambria text fonts definition (SJTUThesis)}
%<textfont&cambria>\ProvidesExplFile{sjtu-text-font-cambria.def}
%<textfont&times>  {Times text fonts definition (SJTUThesis)}
%<textfont&times>\ProvidesExplFile{sjtu-text-font-times.def}
%<mathfont&termes>  {Termes math fonts definition (SJTUThesis)}
%<mathfont&termes>\ProvidesExplFile{sjtu-math-font-termes.def}
%<mathfont&pagella>  {Pagella math fonts definition (SJTUThesis)}
%<mathfont&pagella>\ProvidesExplFile{sjtu-math-font-pagella.def}
%<mathfont&stixtwo>  {STIX Two math fonts definition (SJTUThesis)}
%<mathfont&stixtwo>\ProvidesExplFile{sjtu-math-font-stixtwo.def}
%<mathfont&xits>  {XITS math fonts definition (SJTUThesis)}
%<mathfont&xits>\ProvidesExplFile{sjtu-math-font-xits.def}
%<mathfont&lm>  {Latin Modern math fonts definition (SJTUThesis)}
%<mathfont&lm>\ProvidesExplFile{sjtu-math-font-lm.def}
%<mathfont&newcm>  {New Computer Modern math fonts definition (SJTUThesis)}
%<mathfont&newcm>\ProvidesExplFile{sjtu-math-font-newcm.def}
%<mathfont&cambria>  {Cambria math fonts definition (SJTUThesis)}
%<mathfont&cambria>\ProvidesExplFile{sjtu-math-font-cambria.def}
%<cjkfont&windows>  {Windows CJK fonts definition (SJTUThesis)}
%<cjkfont&windows>\ProvidesExplFile{sjtu-cjk-font-windows.def}
%<cjkfont&mac>  {macOS CJK fonts definition (SJTUThesis)}
%<cjkfont&mac>\ProvidesExplFile{sjtu-cjk-font-mac.def}
%<cjkfont&ubuntu>  {Ubuntu CJK fonts definition (SJTUThesis)}
%<cjkfont&ubuntu>\ProvidesExplFile{sjtu-cjk-font-ubuntu.def}
%<cjkfont&adobe>  {Adobe CJK fonts definition (SJTUThesis)}
%<cjkfont&adobe>\ProvidesExplFile{sjtu-cjk-font-adobe.def}
%<cjkfont&fandol>  {Fandol CJK fonts definition (SJTUThesis)}
%<cjkfont&fandol>\ProvidesExplFile{sjtu-cjk-font-fandol.def}
%<cjkfont&founder>  {Founder CJK fonts definition (SJTUThesis)}
%<cjkfont&founder>\ProvidesExplFile{sjtu-cjk-font-founder.def}
%<!driver>  {\ExplFileDate}{\ExplFileVersion}{\ExplFileDescription}
%</!(driver|install)>
%
%<*driver>
\documentclass{ctxdoc}
\begin{document}
  \DocInput{\jobname.dtx}
  \IndexLayout
  \PrintChanges
  \PrintIndex
\end{document}
%</driver>
% \fi
%
% \GetFileInfo{\jobname.dtx}
%
% \begin{documentation}
%
% \section{介绍}
%
% Hello, world! 你好,世界!
%
% \end{documentation}
%
% \begin{implementation}
%
% \clearpage
% \section{代码实现}
%
%    \begin{macrocode}
%<@@=sjtu_ent>
%<*entry>
\RequirePackage { l3keys2e }
\int_new:N \g_@@_thesis_type_int
\int_new:N \g_@@_thesis_lang_int
%    \end{macrocode}
%
% 定义 |sjtu/diversion| 键值类。
%    \begin{macrocode}
\keys_define:nn { sjtu / diversion }
  {
    type .choice: ,
    type .value_required:n = true ,
    type .choices:nn =
      { doctor, master, bachelor, course }
      { \int_set_eq:NN \g_@@_thesis_type_int \l_keys_choice_int } ,
    type .initial:n = { master } ,
    lang .choice: ,
    lang .value_required:n = true ,
    lang .choices:nn =
      { zh, en }
      { \int_set_eq:NN \g_@@_thesis_lang_int \l_keys_choice_int } ,
    lang .initial:n = { zh } ,
    unknown .code:n = { }
  }
\ProcessKeysOptions { sjtu / diversion }
\int_compare:nTF { \g_@@_thesis_type_int < 3 }
  {
    \int_compare:nTF { \g_@@_thesis_lang_int = 1 }
      { \LoadClassWithOptions { sjtuthesis-postgraduate-zh } }
      { \LoadClassWithOptions { sjtuthesis-postgraduate-en } }
  }
  {
    \int_compare:nTF { \g_@@_thesis_lang_int = 1 }
      { \LoadClassWithOptions { sjtuthesis-undergraduate-zh } }
      { \LoadClassWithOptions { sjtuthesis-undergraduate-en } }
  }
%</entry>
%    \end{macrocode}
%
%    \begin{macrocode}
%<@@=sjtu>
%<*class>
%    \end{macrocode}
%
% \section{实现细节}
%
% 本模板使用 \LaTeX3 语法编写,依赖 \pkg{expl3} 环境,
% 并需调用 \pkg{l3packages} 中的相关宏包。
%
% \subsection{准备}
%
% 检查 \LaTeXiii{} 编程环境。
%    \begin{macrocode}
\RequirePackage { xparse, xtemplate, l3keys2e }
\clist_map_inline:nn { expl3, xparse, xtemplate, l3keys2e }
  {
    \@ifpackagelater {#1} { 2018/05/12 }
      { } { \msg_error:nnn { sjtuthesis } { l3-too-old } {#1} }
  }
\msg_new:nnn { sjtuthesis } { l3-too-old }
  {
    Package~ "#1"~ is~ too~ old. \\\\
    Please~ update~ an~ up-to-date~ version~ of~ the~ bundles \\
    "l3kernel"~ and~ "l3packages"~ using~ your~ TeX~ package \\
    manager~ or~ from~ CTAN.
  }
%    \end{macrocode}
%
% 目前 \cls{sjtuthesis} 仅支持 \XeTeX{} 和 \LuaTeX{}。
%    \begin{macrocode}
\sys_if_engine_xetex:F
  {
    \sys_if_engine_luatex:F
      {
        \msg_fatal:nnx { sjtuthesis } { unsupported-engine }
          { \c_sys_engine_str }
      }
  }
\msg_new:nnn { sjtuthesis } { unsupported-engine }
  {
    The~ sjtuthesis~ class~ requires~ either~ XeTeX~ or~ LuaTeX. \\\\
    "#1"~ is~ not~ supported~ at~ present.~ You~ must~ change \\
    your~ typesetting~ engine~ to~ "xelatex"~ or~ "lualatex".
  }
%    \end{macrocode}
%
% \subsubsection{内部变量}
%
% \begin{variable}{\g_@@_thesis_type_tl,\g_@@_thesis_type_int}
% 论文类型。
%    \begin{macrocode}
\tl_new:N  \g_@@_thesis_type_tl
\int_new:N \g_@@_thesis_type_int
%    \end{macrocode}
% \end{variable}
%
% \begin{variable}{\g_@@_lang_tl}
% 论文语言。
%    \begin{macrocode}
\tl_new:N  \g_@@_lang_tl
%    \end{macrocode}
% \end{variable}
%
% \begin{variable}{\g_@@_font_size_tl,\g_@@_line_spread_fp}
% 字号大小与行距倍数。
%    \begin{macrocode}
\tl_new:N \g_@@_font_size_tl
\fp_new:N \g_@@_line_spread_fp
%    \end{macrocode}
% \end{variable}
%
%    \begin{macrocode}
\fp_const:Nn \c_@@_xiaosi_line_spread_fp { 20 / ( 12   * 1.2 ) }
\fp_const:Nn \c_@@_wuhao_line_spread_fp  { 17 / ( 10.5 * 1.2 ) }
%    \end{macrocode}
%
% \begin{variable}
%   {\g_@@_text_font_tl,\g_@@_math_font_tl,\g_@@_cjk_font_tl}
% 字体配置。
%    \begin{macrocode}
\tl_new:N \g_@@_text_font_tl
\tl_new:N \g_@@_math_font_tl
\tl_new:N \g_@@_cjk_font_tl
\tl_new:N \g_@@_save_encodingdefault_tl
\tl_new:N \g_@@_save_rmdefault_tl
\tl_new:N \g_@@_save_sfdefault_tl
\tl_new:N \g_@@_save_ttdefault_tl
%    \end{macrocode}
% \end{variable}
%
% \begin{variable}{\g_@@_unimath_bool}
% 是否偏好使用 \pkg{unicode-math}。
%    \begin{macrocode}
\bool_new:N \g_@@_unimath_bool
%    \end{macrocode}
% \end{variable}
%
% \begin{variable}
%  {\g_@@_ctex_options_clist,\g_@@_hyperref_options_clist}
% 分别保存由 \cls{sjtuthesis} 传入 \cls{ctex} 文档类和
% \pkg{hyperref} 宏包的选项列表。
%    \begin{macrocode}
\clist_new:N \g_@@_ctex_options_clist
\clist_new:N \g_@@_hyperref_options_clist
%    \end{macrocode}
% \end{variable}
%
% \subsubsection{内部函数}
%
% \begin{macro}{\@@_define_name:nn,\@@_define_name:nnn}
% 用来定义默认名称的辅助函数。
%    \begin{macrocode}
\cs_new_protected:Npn \@@_define_name:nn #1#2
  { \tl_const:cn { c_@@_name_ #1 _zh_tl } {#2} }
\cs_new_protected:Npn \@@_define_name:nnn #1#2#3
  {
    \tl_const:cn { c_@@_name_ #1 _zh_tl } {#2}
    \tl_const:cn { c_@@_name_ #1 _en_tl } {#3}
    \tl_const:cx { c_@@_name_ #1 _tl }
      { \use:c { c_@@_name_ #1 _ \g_@@_lang_tl _tl } }
  }
\cs_new_protected:Npn \@@_define_name_list:Nnn  #1#2#3
  {
    \clist_const:cn { c_@@_name_ #2 _zh_clist } {#3}
    \tl_const:cx { c_@@_name_ #2 _zh_tl }
      { \clist_item:cn { c_@@_name_ #2 _zh_clist } {#1} }
  }
\cs_new_protected:Npn \@@_define_name_list:Nnnn #1#2#3#4
  {
    \clist_const:cn { c_@@_name_ #2 _zh_clist } {#3}
    \clist_const:cn { c_@@_name_ #2 _en_clist } {#4}
    \tl_const:cx { c_@@_name_ #2 _zh_tl }
      { \clist_item:cn { c_@@_name_ #2 _zh_clist } {#1} }
    \tl_const:cx { c_@@_name_ #2 _en_tl }
      { \clist_item:cn { c_@@_name_ #2 _en_clist } {#1} }
    \tl_const:cx { c_@@_name_ #2 _tl }
      { \use:c { c_@@_name_ #2 _ \g_@@_lang_tl _tl } }
  }
%    \end{macrocode}
% \end{macro}
%
% \begin{macro}{\@@_patch_cmd:Nnn,\@@_appto_cmd:Nn}
% 补丁工具,来自 \pkg{ctexpatch} 宏包。
%    \begin{macrocode}
\cs_new_protected:Npn \@@_patch_cmd:Nnn #1#2#3
  {
    \ctex_patch_cmd_once:NnnnTF #1 { } {#2} {#3}
      { } { \ctex_patch_failure:N #1 }
  }
\cs_new_protected:Npn \@@_appto_cmd:Nn #1#2
  {
    \ctex_appto_cmd:NnnTF #1 { } {#2}
      { } { \ctex_patch_failure:N #1 }
  }
%    \end{macrocode}
% \end{macro}
%
% \subsection{选项处理}
%
% 定义 |sjtu/option| 键值类。
%    \begin{macrocode}
\keys_define:nn { sjtu / option }
  {
%    \end{macrocode}
%
%    \begin{macrocode}
    type .choice: ,
    type .value_required:n = true ,
    type .choices:nn =
      { doctor, master, bachelor, course }
      { \int_gset_eq:NN \g_@@_thesis_type_int \l_keys_choice_int } ,
%<pg>    type .initial:n = { master   } ,
%<ug>    type .initial:n = { bachelor } ,
%    \end{macrocode}
%
%
%    \begin{macrocode}
    lang .choice: ,
    lang .value_required:n = true ,
    lang .choices:nn =
      { zh, en }
      { \tl_gset_eq:NN \g_@@_lang_tl \l_keys_choice_tl } ,
    lang .initial:n = { zh } ,
%    \end{macrocode}
%
% 字号大小。
%    \begin{macrocode}
    zihao .choice: ,
    zihao .value_required:n = true ,
    zihao / -4 .code:n =
      {
        \tl_gset:Nn    \g_@@_font_size_tl   { -4 }
        \fp_gset_eq:NN \g_@@_line_spread_fp \c_@@_xiaosi_line_spread_fp
      } ,
    zihao /  5 .code:n =
      {
        \tl_gset:Nn    \g_@@_font_size_tl   {  5 }
        \fp_gset_eq:NN \g_@@_line_spread_fp \c_@@_wuhao_line_spread_fp
      } ,
%<pg>    zihao .initial:n = { -4 } ,
%<ug>    zihao .initial:n = {  5 } ,
%    \end{macrocode}
%
% 字体配置。
%    \begin{macrocode}
    text-font .tl_gset:N = \g_@@_text_font_tl ,
    text-font .initial:n = { termes } ,
    math-font .tl_gset:N = \g_@@_math_font_tl ,
    cjk-font  .tl_gset:N = \g_@@_cjk_font_tl ,
%    \end{macrocode}
%
% \pkg{unicode-math}。
%    \begin{macrocode}
    unimath .bool_gset:N = \g_@@_unimath_bool ,
    unimath .initial:n = false ,
%    \end{macrocode}
%
% 处理未知选项。
%    \begin{macrocode}
    unknown .code:n = { \msg_error:nn { sjtuthesis } { unknown-option } }
  }
\msg_new:nnn { sjtuthesis } { unknown-option }
  { Class~ option~ "\l_keys_key_tl"~ is~ unknown. }
%    \end{macrocode}
%
% 将文档类选项传给 |sjtu/option|。
%    \begin{macrocode}
\ProcessKeysOptions { sjtu / option }
%    \end{macrocode}
%
% 载入配置文件。
%    \begin{macrocode}
\file_input:n { sjtuthesis-name.cfg }
%    \end{macrocode}
%
% \subsection{载入宏包、文档类}
%
% 将选项传入 \pkg{ctex} 文档类。
%    \begin{macrocode}
\PassOptionsToClass
  {
    UTF8,
%<en>    scheme = plain,
    fontset = none,
    zihao = \g_@@_font_size_tl,
    linespread = \g_@@_line_spread_fp,
    \g_@@_ctex_options_clist
  }
%<pg>  { ctexbook }
%<ug>  { ctexrep  }
%    \end{macrocode}
%
% 传入各宏包选项。
%    \begin{macrocode}
\clist_map_inline:nn
  {
    { no-math         } { fontspec     },
    { titles          } { tocloft      },
    { perpage, bottom } { footmisc     },
    { list = off      } { bicaption    },
    { warnings-off = 
      { 
        mathtools-overbracket,
        mathtools-colon
      } 
    }                   { unicode-math }
  }
  { \PassOptionsToPackage #1 }
%    \end{macrocode}
%
% 载入 \cls{ctexbook} 文档类。
% 在使用 \XeLaTeX{} 编译时,\cls{ctexbook} 的底层将调用 \pkg{xeCJK}
% 宏包;而在使用 \LuaLaTeX{} 编译时,则将调用 \pkg{LuaTeX-ja} 宏包。
% 两种情况下 \cls{ctexbook} 均会调用 \pkg{fontspec} 宏包。
%    \begin{macrocode}
%<pg>\LoadClass { ctexbook }
%<ug>\LoadClass { ctexrep  }
%    \end{macrocode}
%
% 载入各宏包。
%    \begin{macrocode}
\RequirePackage
  {
    mathtools,
    geometry,
    fancyhdr, 
    tocloft,
    footmisc,
    graphicx,
    caption,
    bicaption,
    subcaption,
    xcolor
  }
%    \end{macrocode}
%
%    \begin{macrocode}
\msg_new:nnn { sjtuthesis } { invalid-font }
  {
    Invalid~ value~ `#1-font~ =~ \use:c{ g_@@_ #1 _font_tl }~ '! \\\\
    Using~ `#2'~ instead.
  }
\tl_if_empty:NT \g_@@_math_font_tl
  { \tl_gset_eq:NN \g_@@_math_font_tl \g_@@_text_font_tl }
%    \end{macrocode}
% 由于 ctex 自带字体描述文件对于 AutoFakeBold 的支持不太好,SJTUThesis 改写了 ctex
% 的字体描述文件,使得 LaTeX 可以使用系统字体生成加粗效果看得过去的文本。由此满足了学校
% 对于论文表题图题加粗的规定。
%    \begin{macrocode}
\tl_if_empty:NT \g_@@_cjk_font_tl
  {
    \fontspec_font_if_exist:nTF { SimSun }
      { \tl_gset:Nn \g_@@_cjk_font_tl { windows } }
      {
        \ctex_if_platform_macos:TF
          { \tl_gset:Nn \g_@@_cjk_font_tl { mac } }
          {
            \fontspec_font_if_exist:nTF { Noto~Serif~CJK~SC }
              { \tl_gset:Nn \g_@@_cjk_font_tl { ubuntu } }
              { \tl_gset:Nn \g_@@_cjk_font_tl { fandol } }
          }
      }
  }
\cs_new_protected:Npn \@@_load_font:nn #1#2
  {
    \str_if_eq:onF { \use:c{ g_@@_ #1 _font_tl } } { none }
      {
        \ctex_push_file:
          \file_if_exist_input:nF
            { sjtu- #1 -font- \use:c{ g_@@_ #1 _font_tl } .def }
            {
              \msg_warning:nnnn { sjtuthesis } { invalid-font } {#1} {#2}
              \tl_gset:cn { g_@@_ #1 _font_tl } {#2}
              \file_input:n
                { sjtu- #1 -font- \use:c{ g_@@_ #1 _font_tl } .def }
            }
        \ctex_pop_file:
      }
  }
\cs_new_protected:Npn \@@_load_fontset:
  {
    \clist_map_inline:nn
      {
        { math } { termes },
        { text } { termes },
        { cjk  } { fandol }
      }
      { \@@_load_font:nn ##1 }
  }
\@onlypreamble \@@_load_font:nn
\@onlypreamble \@@_load_fontset:
\@@_load_fontset:
%    \end{macrocode}
%
%    \begin{macrocode}
%</class>
%    \end{macrocode}
%
%    \begin{macrocode}
%<*mathfont>
%<termes|pagella|lm>\bool_if:NTF \g_@@_unimath_bool 
%<termes|pagella|lm>  {
%<stixtwo|xits|newcm|cambria>\bool_set_true:N \g_@@_unimath_bool
    \RequirePackage { unicode-math }
%<termes>    \setmathfont { texgyretermes-math.otf }
%<pagella>    \setmathfont { texgyrepagella-math.otf }
%<*stixtwo>
\fontspec_font_if_exist:nTF { STIXTwoMath-Regular.otf }
  {
    \setmathfont { STIXTwoMath-Regular.otf }
      [ StylisticSet = 8 ]
    \setmathfont { STIXTwoMath-Regular.otf }
      [
        range        = {scr, bfscr},
        StylisticSet = 1
      ]
  }
  {
    \setmathfont { STIX2Math.otf }
      [ StylisticSet = 8 ]
    \setmathfont { STIX2Math.otf }
      [
        range        = {scr, bfscr},
        StylisticSet = 1
      ]
  }
%</stixtwo>
%<*xits>
\fontspec_font_if_exist:nTF { XITSMath-Regular.otf }
  {
    \setmathfont { XITSMath-Regular }
      [
        Extension    = .otf,
        BoldFont     = XITSMath-Bold,
        StylisticSet = 8
      ]
    \setmathfont { XITSMath-Regular.otf }
      [
        range        = {cal, bfcal},
        StylisticSet = 1
      ]
  }
  {
    \setmathfont { xits-math }
      [
        Extension    = .otf,
        BoldFont     = *bold,
        StylisticSet = 8
      ]
    \setmathfont { xits-math.otf }
      [
        range        = {cal, bfcal},
        StylisticSet = 1
      ]
  }
%</xits>
%<lm>    \setmathfont { latinmodern-math.otf }
%<*newcm>
\setmathfont { NewCMMath-Book.otf }
  [ StylisticSet = 2 ]
\setmathfont { NewCMMath-Book.otf }
  [
    range        = {scr, bfscr},
    StylisticSet = 1
  ]
%</newcm>
%<cambria>\setmathfont { Cambria~Math }
%</mathfont>
%<*termes|pagella>
%<mathfont>    \setmathrm
%<textfont>    \setmainfont
%<termes>      { texgyretermes }
%<pagella>      { texgyrepagella }
      [
        Extension      = .otf,
        UprightFont    = *-regular,
        BoldFont       = *-bold,
        ItalicFont     = *-italic,
        BoldItalicFont = *-bolditalic,
      ]
%</termes|pagella>
%<*stixtwo>
\fontspec_font_if_exist:nTF { STIXTwoText-Regular.otf }
  {
%<mathfont>    \setmathrm
%<textfont>    \setmainfont
      { STIXTwoText }
      [
        Extension      = .otf,
        UprightFont    = *-Regular,
        BoldFont       = *-Bold,
        ItalicFont     = *-Italic,
        BoldItalicFont = *-BoldItalic
      ]
  }
  {
%<mathfont>    \setmathrm
%<textfont>    \setmainfont
      { STIX2Text }
      [
        Extension      = .otf,
        UprightFont    = *-Regular,
        BoldFont       = *-Bold,
        ItalicFont     = *-Italic,
        BoldItalicFont = *-BoldItalic
      ]
  }
%</stixtwo>
%<*xits>
\fontspec_font_if_exist:nTF { XITS-Regular.otf }
  {
%<mathfont>    \setmathrm
%<textfont>    \setmainfont
      { XITS }
      [
        Extension      = .otf,
        UprightFont    = *-Regular,
        BoldFont       = *-Bold,
        ItalicFont     = *-Italic,
        BoldItalicFont = *-BoldItalic
      ]
  }
  {
%<mathfont>    \setmathrm
%<textfont>    \setmainfont
      { xits }
      [
        Extension      = .otf,
        UprightFont    = *-regular,
        BoldFont       = *-bold,
        ItalicFont     = *-italic,
        BoldItalicFont = *-bolditalic
      ]
  }
%</xits>
%<*termes|pagella|stixtwo|xits>
%<mathfont>    \setmathsf
%<textfont>    \setsansfont
      { texgyreheros }
      [
        Extension      = .otf,
        UprightFont    = *-regular,
        BoldFont       = *-bold,
        ItalicFont     = *-italic,
        BoldItalicFont = *-bolditalic
      ]
%<mathfont>    \setmathtt
%<textfont>    \setmonofont
      { texgyrecursor }
      [
        Extension      = .otf,
        UprightFont    = *-regular,
        BoldFont       = *-bold,
        ItalicFont     = *-italic,
        BoldItalicFont = *-bolditalic,
        Scale          = MatchLowercase,
        Ligatures      = CommonOff
      ]
%</termes|pagella|stixtwo|xits>
%<*lm>
%<mathfont>    \setmathrm
%<textfont>    \setmainfont
      { lmroman10 }
      [
        Extension      = .otf,
        UprightFont    = *-regular,
        BoldFont       = *-bold,
        ItalicFont     = *-italic,
        BoldItalicFont = *-bolditalic
      ]
%<mathfont>    \setmathsf
%<textfont>    \setsansfont
      { lmsans10 }
      [
        Extension      = .otf,
        UprightFont    = *-regular,
        BoldFont       = *-bold,
        ItalicFont     = *-oblique,
        BoldItalicFont = *-boldoblique
      ]
%<mathfont>    \setmathtt
%<textfont>    \setmonofont
      { lmmonolt10 }
      [
        Extension      = .otf,
        UprightFont    = *-regular,
        BoldFont       = *-bold,
        ItalicFont     = *-oblique,
        BoldItalicFont = *-boldoblique
      ]
%</lm>
%<*newcm>
%<mathfont>\setmathrm
%<textfont>\setmainfont
  { NewCM10 }
  [
    Extension      = .otf,
    UprightFont    = *-Book,
    BoldFont       = *-Bold,
    ItalicFont     = *-BookItalic,
    BoldItalicFont = *-BoldItalic
  ]
%<mathfont>\setmathsf
%<textfont>\setsansfont
  { NewCMSans10 }
  [
    Extension      = .otf,
    UprightFont    = *-Book,
    BoldFont       = *-Bold,
    ItalicFont     = *-BookOblique,
    BoldItalicFont = *-BoldOblique
  ]
%<mathfont>\setmathtt
%<textfont>\setmonofont
  { NewCMMono10 }
  [
    Extension      = .otf,
    UprightFont    = *-Book,
    BoldFont       = *-Bold,
    ItalicFont     = *-BookItalic,
    BoldItalicFont = *-BoldOblique
  ]
%</newcm>
%<*cambria>
%<*mathfont>
\setmathrm { Cambria }
\setmathsf { Calibri }
\setmathtt { Consolas } [ Scale = MatchLowercase ]
%</mathfont>
%<*textfont>
\setmainfont { Cambria }
\setsansfont { Calibri }
\setmonofont { Consolas } [ Scale = MatchLowercase ]
%</textfont>
%</cambria>
%<*times>
\setmainfont { Times~New~Roman } [ Ligatures = Rare ]
\setsansfont { Arial }
\setmonofont { Courier~New } [ Scale = MatchLowercase ]
%</times>
%<*mathfont&(termes|pagella|lm)>
  }
  {
%<lm>    \RequirePackage { amssymb, upgreek }
    \tl_set_eq:NN \g_@@_save_encodingdefault_tl \encodingdefault
    \tl_set_eq:NN \g_@@_save_rmdefault_tl \rmdefault
    \tl_set_eq:NN \g_@@_save_sfdefault_tl \sfdefault
    \tl_set_eq:NN \g_@@_save_ttdefault_tl \ttdefault
    \RequirePackage [ T1 ] { fontenc }
%<termes>    \tl_set:Nn \rmdefault { ntxtlf }
%<pagella>    \tl_set:Nn \rmdefault { zpltlf }
%<termes|pagella>    \tl_set:Nn \sfdefault { qhv }
%<termes|pagella>    \tl_set:Nn \ttdefault { ntxtt }
%<termes>    \RequirePackage [ upint ] { newtxmath }
%<pagella>    \RequirePackage [ upint ] { newpxmath }
%<lm>    \RequirePackage { lmodern }
    \tl_set_eq:NN \encodingdefault \g_@@_save_encodingdefault_tl
    \tl_set_eq:NN \rmdefault \g_@@_save_rmdefault_tl
    \tl_set_eq:NN \sfdefault \g_@@_save_sfdefault_tl
    \tl_set_eq:NN \ttdefault \g_@@_save_ttdefault_tl
    \RequirePackage { bm }
  }
%</mathfont&(termes|pagella|lm)>
%    \end{macrocode}
%
%    \begin{macrocode}
%<*cjkfont>
%<*windows>
\setCJKmainfont { SimSun   }
  [ AutoFakeBold = 3, ItalicFont = KaiTi ]
\setCJKsansfont { SimHei   } [ BoldFont = * ]
\setCJKmonofont { FangSong }
\setCJKfamilyfont { zhsong } { SimSun   }
  [ AutoFakeBold = 3, ItalicFont = KaiTi ]
\setCJKfamilyfont { zhhei  } { SimHei   } [ BoldFont = * ]
\setCJKfamilyfont { zhkai  } { KaiTi    }
\setCJKfamilyfont { zhfs   } { FangSong }
%</windows>
%<*mac>
\setCJKmainfont { Songti~SC  }
  [
    UprightFont    = *~Light,
    BoldFont       = *~Bold,
    ItalicFont     = Kaiti~SC~Regular,
    BoldItalicFont = Kaiti~SC~Bold
  ]
\setCJKsansfont { Heiti~SC   }
  [
    UprightFont    = *~Medium,
    BoldFont       = *~Medium
  ]
\setCJKmonofont { STFangsong }
\setCJKfamilyfont { zhsong } { Songti~SC  }
  [
    UprightFont    = *~Light,
    BoldFont       = *~Bold
  ]
\setCJKfamilyfont { zhhei  } { Heiti~SC   }
  [
    UprightFont    = *~Medium,
    BoldFont       = *~Medium
  ]
\setCJKfamilyfont { zhfs   } { STFangsong }
\setCJKfamilyfont { zhkai  } { Kaiti~SC   }
  [
    UprightFont    = *~Regular,
    BoldFont       = *~Bold
  ]
%</mac>
%<*ubuntu>
\setCJKmainfont { Noto~Serif~CJK~SC     }
  [
    UprightFont = *~Light,
    BoldFont    = *~Bold,
    ItalicFont  = AR~PL~UKai~CN
  ]
\setCJKsansfont { Noto~Sans~CJK~SC      }
  [
    UprightFont = *~Medium,
    BoldFont    = *~Medium
  ]
\setCJKmonofont { Noto~Sans~Mono~CJK~SC }
\setCJKfamilyfont { zhsong } { Noto~Serif~CJK~SC }
  [
    UprightFont = *~Light,
    BoldFont    = *~Bold,
    ItalicFont  = AR~PL~UKai~CN
  ]
\setCJKfamilyfont { zhhei  } { Noto~Sans~CJK~SC  }
  [
    UprightFont = *~Medium,
    BoldFont    = *~Medium
  ]
\setCJKfamilyfont { zhkai  } { AR~PL~UKai~CN     }
%</ubuntu>
%<*adobe>
\setCJKmainfont { AdobeSongStd-Light       }
  [ AutoFakeBold = 3, ItalicFont = AdobeKaitiStd-Regular ]
\setCJKsansfont { AdobeHeitiStd-Regular    } [ BoldFont = * ]
\setCJKmonofont { AdobeFangsongStd-Regular }
\setCJKfamilyfont { zhsong } { AdobeSongStd-Light       }
  [ AutoFakeBold = 3, ItalicFont = AdobeKaitiStd-Regular ]
\setCJKfamilyfont { zhhei  } { AdobeHeitiStd-Regular    } [ BoldFont = * ]
\setCJKfamilyfont { zhfs   } { AdobeFangsongStd-Regular }
\setCJKfamilyfont { zhkai  } { AdobeKaitiStd-Regular    }
%</adobe>
%<*fandol>
\setCJKmainfont { FandolSong }
  [
    Extension   = .otf,
    UprightFont = *-Regular,
    BoldFont    = *-Bold,
    ItalicFont  = FandolKai-Regular
  ]
\setCJKsansfont { FandolHei  }
  [
    Extension   = .otf,
    UprightFont = *-Regular,
    BoldFont    = *-Regular,
  ]
\setCJKmonofont { FandolFang }
  [
    Extension   = .otf,
    UprightFont = *-Regular,
  ]
\setCJKfamilyfont { zhsong } { FandolSong }
  [
    Extension   = .otf,
    UprightFont = *-Regular,
    BoldFont    = *-Bold
  ]
\setCJKfamilyfont { zhhei  } { FandolHei  }
  [
    Extension   = .otf,
    UprightFont = *-Regular,
    BoldFont    = *-Regular
  ]
\setCJKfamilyfont { zhfs   } { FandolFang }
  [
    Extension   = .otf,
    UprightFont = *-Regular
  ]
\setCJKfamilyfont { zhkai  } { FandolKai  }
  [
    Extension   = .otf,
    UprightFont = *-Regular
  ]
%</fandol>
%<*founder>
\setCJKmainfont { FZShuSong-Z01  } 
  [ AutoFakeBold = 3, ItalicFont = FZKai-Z03 ]
\setCJKsansfont { FZHei-B01      } [ BoldFont = * ]
\setCJKmonofont { FZFangSong-Z02 }
\setCJKfamilyfont { zhsong } { FZShuSong-Z01  }
  [ AutoFakeBold = 3, ItalicFont = FZKai-Z03 ]
\setCJKfamilyfont { zhhei  } { FZHei-B01      } [ BoldFont = * ]
\setCJKfamilyfont { zhkai  } { FZKai-Z03      }
\setCJKfamilyfont { zhfs   } { FZFangSong-Z02 }
%</founder>
\NewDocumentCommand \songti   { } { \CJKfamily { zhsong  } }
\NewDocumentCommand \heiti    { } { \CJKfamily { zhhei   } }
%<!ubuntu>\NewDocumentCommand \fangsong { } { \CJKfamily { zhfs    } }
\NewDocumentCommand \kaishu   { } { \CJKfamily { zhkai   } }
%</cjkfont>
%    \end{macrocode}
%
%    \begin{macrocode}
%<*class>
%    \end{macrocode}
%
%    \begin{macrocode}
\geometry
  {
    paper      = a4paper,
%<*pg>
    top        = 3.5 cm,
    bottom     = 4.0 cm,
    left       = 3.3 cm,
    right      = 2.8 cm,
    headheight = 1.0 cm,
    headsep    = 0.5 cm,
    footskip   = 1.0 cm
%</pg>
%<*ug>
    vmargin    = 2.54 cm,
    hmargin    = 3.18 cm,
    headheight = 15 bp
%</ug>
  }
%    \end{macrocode}
%
% \pkg{ctex} 宏包使用 \opt{heading} 选项后,会把页面格式设置为 |headings|。
% 因此必须在 \pkg{ctex} 调用之后重新设置 \cs{pagestyle} 为 |fancy|。
%    \begin{macrocode}
\pagestyle { fancy }
%    \end{macrocode}
%
%    \begin{macrocode}
\ctex_set:nn { chapter }
  {
    fixskip     = true ,
    beforeskip  = 24 bp ,
    afterskip   = 18 bp ,
    lofskip     = \z@ ,
    lotskip     = \z@ ,
    format      = \zihao { 3 } \bfseries \heiti \centering ,
    nameformat  = ,
    titleformat = ,
    aftername   = \quad ,
    afterindent = true
  }
\ctex_set:nn { section }
  {
    beforeskip  = 24 bp ,
    afterskip   =  6 bp ,
    format      = \zihao { 4 } \bfseries \heiti ,
    afterindent = true
  }
\ctex_set:nn { subsection }
  {
    beforeskip  = 12 bp ,
    afterskip   =  6 bp ,
    format      = \zihao { -4 } \bfseries \heiti ,
    afterindent = true
  }
\ctex_set:nn { subsubsection }
  {
%<ug>    name        = { ( , ) } ,
%<ug>    number      = \arabic { subsubsection } ,
    beforeskip  =  6 bp ,
    afterskip   =  6 bp ,
    format      = \zihao { -4 } \normalfont ,
    afterindent = true
  }
\ctex_set:nn { paragraph }
  { afterindent = true }
\ctex_set:nn { subparagraph }
  { afterindent = true }
\ctex_if_autoindent_touched:F
  { \ctex_set:n { autoindent = true } }
\@@_patch_cmd:Nnn \chapter
  { \thispagestyle } { \@gobble }
%    \end{macrocode}
%
% 设置目录目录格式。
%    \begin{macrocode}
%<zh>\tl_set:Nn \cftdot { \textperiodcentered }
\tl_set:Nn \cftdotsep { 1 }
%<ug>\dim_zero:N \cftbeforechapskip
\cs_set:Npn \cftchappagefont { \normalfont }
\cs_set:Npn \cftchapleader { \normalfont \cftdotfill { \cftdotsep } }
%<pg>\cs_set:Npn \cftchapfont { \bfseries \heiti }
%<ug>\cs_set:Npn \cftchapfont { \normalfont }
%    \end{macrocode}
%
%    \begin{macrocode}
\fancyhf { }
\fancyhead[RE,LO]{\c_@@_name_univ_zh_tl \c_@@_name_degree_level_zh_tl 论文}
\fancyhead[RO,LE]{\leftmark}
\NewDocumentCommand \sjtusetup { } { \keys_set:nn { sjtu } }
\keys_define:nn { sjtu }
  {
    name                  .meta:nn = { sjtu / name } {#1} ,
    name / contents      .tl_set:N = \contentsname ,
    name / listfigure    .tl_set:N = \listfigurename ,
    name / listtable     .tl_set:N = \listtablename ,
    name / listalgorithm .tl_set:N = \listalgorithmname ,
    name / figure        .tl_set:N = \figurename ,
    name / table         .tl_set:N = \tablename ,
    name / abstract      .tl_set:N = \abstractname ,
    name / index         .tl_set:N = \indexname ,
    name / appendix      .tl_set:N = \appendixname ,
    name / proof         .tl_set:N = \proofname ,
    name / bib           .tl_set:N = \bibname
  }
%</class>
%    \end{macrocode}
%
%    \begin{macrocode}
%<*name>
\@@_define_name:nnn { univ }
  { 上海交通大学 }
  { Shanghai~Jiao~Tong~University }
\@@_define_name_list:Nnnn \g_@@_thesis_type_int { degree_level }
  { 博士  , 硕士  , 学士    ,  }
  { Doctor, Master, Bachelor,  }
\tl_const:Nn \c_@@_orig_decl_text_tl
  {
    本人郑重声明:所呈交的学位论文,是本人在导师的指导下,独立进行研究工
    作所取得的成果。除文中已经注明引用的内容外,本论文不包含任何其他个人
    或集体已经发表或撰写过的作品成果。对本文的研究做出重要贡献的个人和集
    体,均已在文中以明确方式标明。本人完全意识到本声明的法律结果由本人承
    担。
  }
\tl_const:Nn \c_@@_auth_decl_text_tl
  {
    本学位论文作者完全了解学校有关保留、使用学位论文的规定,同意学校保留
    并向国家有关部门或机构送交论文的复印件和电子版,允许论文被查阅和借
    阅。
  }
\clist_map_inline:nn
  {
    { appendix      } { 附录         } { Appendix           } ,
    { contents      } { 目 \quad 录  } { Contents           } ,
    { listfigure    } { 插图索引     } { List of Figures    } ,
    { listtable     } { 表格索引     } { List of Tables     } ,
    { listalgorithm } { 算法索引     } { List of Algorithms } ,
    { figure        } { 图           } { Figure             } ,
    { figure_second } { Figure       } { 图                 } ,
    { table         } { 表           } { Table              } ,
    { table_second  } { Table        } { 表                 } ,
    { algorithm     } { 算法         } { Algorithm          } ,
    { nom           } { 符号对照表   } { Abbreviation       } ,
    { abbr          } { 缩略语对照表 } { Nomenclature       } ,
    { bib           } { 参考文献     } { Bibliography       } ,
    { index         } { 索 \quad 引  } { Index              } ,
    { ack           } { 致 \quad 谢  } { Acknowledgements   } ,
    { resume        } { 个人简历     } { Publications       } ,
    { digest        } { 大摘要       } { Digest             } ,
    { publications  }
      { 攻读学位期间发表(或录用)的学术论文 } { Publications          } ,
    { achievements  }
      { 攻读学位期间获得的科研成果           } { Research Achievements }
  }
  { \@@_define_name:nnn #1 }
\clist_map_inline:nn
  { 
    { assumption  } { 假设 } { Assumption  } ,
    { axiom       } { 公理 } { Axiom       } ,
    { conjecture  } { 猜想 } { Conjecture  } ,
    { corollary   } { 推论 } { Corollary   } ,
    { definition  } { 定义 } { Definition  } ,
    { example     } { 例   } { Example     } ,
    { exercise    } { 练习 } { Exercise    } ,
    { lemma       } { 引理 } { Lemma       } ,
    { problem     } { 问题 } { Problem     } ,
    { proof       } { 证明 } { Proof       } ,
    { proposition } { 命题 } { Proposition } ,
    { remark      } { 注   } { Remark      } ,
    { solution    } { 解   } { Solution    } ,
    { theorem     } { 定理 } { Theorem     } ,
  }
  { \@@_define_name:nnn #1 }
%    \end{macrocode}
%
%    \begin{macrocode}
%</name>
%    \end{macrocode}
%
% \end{implementation}
%
% \Finale
%
\endinput
