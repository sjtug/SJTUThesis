% !TEX root = ./main.tex

\sjtusetup{
  %
  %******************************
  % 注意:
  %   1. 配置里面不要出现空行
  %   2. 不需要的配置信息可以删除
  %******************************
  %
  % 信息录入
  %
  info = {%
    %
    % 标题
    %
    title           = {上海交通大学学位论文 \LaTeX{} 模板示例文档},
    title*          = {A Sample Document for \LaTeX-based SJTU Thesis Template},
    %
    % 标题页标题
    %   可使用“\\”命令手动控制换行
    %
    % display-title   = {上海交通大学学位论文\\ \LaTeX{} 模板示例文档},
    % display-title*  = {A Sample Document \\ for \LaTeX-based SJTU Thesis Template},
    %
    % 关键词
    %
    keywords        = {上海交大, 饮水思源, 爱国荣校},
    keywords*       = {SJTU, master thesis, XeTeX/LaTeX template},
    %
    % 姓名
    %
    author          = {某\quad{}某},
    author*         = {Mo Mo},
    %
    % 指导教师
    %
    supervisor      = {某某教授},
    supervisor*     = {Prof. Mou Mou},
    %
    % 副指导教师
    %
    % assoc-supervisor  = {某某教授},
    % assoc-supervisor* = {Prof. Uom Uom},
    %
    % 学号
    %
    id              = {0010900990},
    %
    % 学位
    %   本科生不需要填写
    %
    degree          = {工学硕士},
    degree*         = {Master of Engineering},
    %
    % 专业
    %
    major           = {某某专业},
    major*          = {A Very Important Major},
    %
    % 所属院系
    %
    department      = {某某系},
    department*     = {Depart of XXX},
    %
    % 答辩日期
    %   使用 ISO 格式 (yyyy-mm-dd);默认为当前时间
    %
    % date            = {2014-12-17},
    %
    % 资助基金
    %
    % fund  = {
    %           {国家 973 项目 (No. 2025CB000000)},
    %           {国家自然科学基金 (No. 81120250000)},
    %         },
    % fund* = {
    %           {National Basic Research Program of China (Grant No. 2025CB000000)},
    %           {National Natural Science Foundation of China (Grant No. 81120250000)},
    %         },
  },
  %
  % 风格设置
  %
  style = {%
    %
    % 论文标题页 logo 颜色 (red/blue/black)
    %
    % title-logo-color = black,
  },
  %
  % 名称设置
  %
  name = {
    % bib           = {References},
    % ack           = {谢\hspace{\ccwd}辞},
    % achv          = {攻读学位期间完成的论文},
  },
}

% 使用 BibLaTeX 处理参考文献
%   biblatex-gb7714-2015 常用选项
%     gbnamefmt=lowercase     姓名大小写由输入信息确定
%     gbpub=false             禁用出版信息缺失处理
\usepackage[backend=biber,style=gb7714-2015]{biblatex}
% 文献表字体
% \renewcommand{\bibfont}{\zihao{-5}}
% 文献表条目间的间距
\setlength{\bibitemsep}{0pt}
% 导入参考文献数据库
\addbibresource{bibdata/thesis.bib}

% 脚注格式
\usepackage[perpage,bottom,hang]{footmisc}

% 定义图片文件目录与扩展名
\graphicspath{{figures/}}
\DeclareGraphicsExtensions{.pdf,.eps,.png,.jpg,.jpeg}

% 确定浮动对象的位置,可以使用 [H],强制将浮动对象放到这里(可能效果很差)
% \usepackage{float}

% 固定宽度的表格
% \usepackage{tabularx}

% 使用三线表:toprule,midrule,bottomrule。
\usepackage{booktabs}

% 表格中支持跨行
\usepackage{multirow}

% 表格中数字按小数点对齐
\usepackage{dcolumn}
\newcolumntype{d}[1]{D{.}{.}{#1}}

% 使用长表格
\usepackage{longtable}

% 附带脚注的表格
\usepackage{threeparttable}

% 附带脚注的长表格
\usepackage{threeparttablex}

% 算法环境宏包
\usepackage[ruled,vlined,linesnumbered]{algorithm2e}
% \usepackage{algorithm, algorithmicx, algpseudocode}

% 代码环境宏包
\usepackage{listings}
\lstnewenvironment{codeblock}[1][]%
  {\lstset{style=lstStyleCode,#1}}{}

% 物理科学和技术中使用的数学符号,定义了 \qty 命令,与 siunitx 3.0 有冲突
% \usepackage{physics}

% 直立体数学符号
\providecommand{\dd}{\mathop{}\!\mathrm{d}}
\providecommand{\ee}{\mathrm{e}}
\providecommand{\ii}{\mathrm{i}}
\providecommand{\jj}{\mathrm{j}}

% 国际单位制宏包
\usepackage{siunitx}[=v2]

% 定理环境宏包
\usepackage{ntheorem}
% \usepackage{amsthm}

% 绘图宏包
\usepackage{tikz}
\usetikzlibrary{shapes.geometric, arrows}

% 一些文档中用到的 logo
\usepackage{hologo}
\providecommand{\XeTeX}{\hologo{XeTeX}}
\providecommand{\BibLaTeX}{\textsc{Bib}\LaTeX}

% 借用 ltxdoc 里面的几个命令方便写文档
\DeclareRobustCommand\cs[1]{\texttt{\char`\\#1}}
\providecommand\pkg[1]{{\sffamily#1}}

% 自定义命令

% E-mail
\newcommand{\email}[1]{\href{mailto:#1}{\texttt{#1}}}

% hyperref 宏包在最后调用
\usepackage{hyperref}

% 自动引用题注更正为中文
\def\equationautorefname{式}
\def\footnoteautorefname{脚注}
\def\itemautorefname{项}
\def\figureautorefname{图}
\def\tableautorefname{表}
\def\partautorefname{篇}
\def\appendixautorefname{附录}
\def\chapterautorefname{章}
\def\sectionautorefname{节}
\def\subsectionautorefname{小节}
\def\subsubsectionautorefname{小节}
\def\paragraphautorefname{段落}
\def\subparagraphautorefname{子段落}
\def\FancyVerbLineautorefname{行}
\def\theoremautorefname{定理}
