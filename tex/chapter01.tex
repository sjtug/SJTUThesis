%%==================================================
%% chapter01.tex for SJTU Master Thesis
%%==================================================

%\bibliographystyle{sjtu2}%[此处用于每章都生产参考文献]
\chapter{这是什么}
\label{chap:what}

这是上海交通大学(非官方)学位论文{\LaTeX}模板,当前版本是{\version}。

最早的一版学位模板是一位热心的物理系同学制作的。
那份模板参考了自动化所学位论文模板,使用了CASthesis.cls文档类,中文字符处理则采用当时最为流行的{\CJKLaTeX}方案。
我根据交大研究生院对学位论文的要求
\footnote{\url{http://www.gs.sjtu.edu.cn/policy/fileShow.ahtml?id=130}}
,结合少量个人审美喜好,完成了一份基本可用的交大{\LaTeX}学位论文模板。
但是,搭建一个{\CJKLaTeX}环境并不简单,单单在Linux下配置环境和添加中文字体,就足够让新手打退堂鼓。
在William Wang的建议下,我开始着手把模板向{\XeTeX}引擎移植。
他完成了最初的移植,多亏了他出色的工作,后续的改善工作也得以顺利进行。

随着我对{\LaTeX}系统认知增加,我又断断续续做了一些完善模板的工作,在原有硕士学位论文模板的基础上完成了交大学士和博士学位论文模板。

现在,交大学位论文模板SJTUTHesis代码在github
\footnote{\url{https://github.com/weijianwen/SJTUThesis}}
上维护。
你可以\href{https://github.com/weijianwen/SJTUThesis/issues}{在github上开issue}
、或者在\href{https://bbs.sjtu.edu.cn/bbsdoc?board=TeX_LaTeX}{水源LaTeX版}发帖来反映遇到的问题。

\section{使用模板}

\subsection{准备工作}
\label{sec:requirements}

要使用这个模板撰写学位论文,需要在\emph{TeX系统}、\emph{中英文字体}、\emph{TeX技能}上有所准备。

\begin{itemize}[noitemsep,topsep=0pt,parsep=0pt,partopsep=0pt]
	\item {\TeX}系统:所使用的{\TeX}系统要支持{\XeTeX}引擎,以2014年以后发布的\emph{完整}CTeX、TeXLive、MacTeX发行版为佳。
	\item 中英文字体:操作系统中需要安装\footnote{在Windows、Mac OS X 以及 Linux 上安装额外的字体,可以参考\href{https://www.searchfreefonts.com/articles/how-to-install-fonts.htm}{“How to install fonts?”}。
}TeX Gyre Termes字体\footnote{\url{http://www.gust.org.pl/projects/e-foundry/tex-gyre/termes}}和四款Adobe中文字体
\footnote{请从合法渠道获得Adobe字体。}:AdobeSongStd、AdobeKaitiStd、AdobeHeitiStd、AdobeFangsongStd。
	\item TeX技能:尽管提供了对模板的必要说明,但这不是一份“{\LaTeX}入门文档”。在使用前请先通读其他入门文档。
	\item 针对Windows用户的额外需求:学位论文模本默认使用GNUMake在命令提示符下构建,建议从Cygwin\footnote{\url{http://cygwin.com}}安装GNU工具集。
\end{itemize}

\subsection{模板选项}
\label{sec:thesisoption}

sjtuthesis提供了一些常用选项,在thesis.tex在导入sjtuthesis模板类时,可以组合使用。
这些选项包括:

\begin{itemize}[noitemsep,topsep=0pt,parsep=0pt,partopsep=0pt]
\item 学位类型:bachelor(学位)、master(硕士)、doctor(博士),是必选项。
\item 中国字体:adobefonts(Adobe中文字体)、winfonts(使用Windows下的中文字体,改选项未在Linux/Mac下测试)。
\item 正文字号:cs4size(小四)、c5size(五号)。
\item 盲审选项:使用review选项后,论文作者、学号、导师姓名、致谢、发表论文和参与项目将被隐去。
\end{itemize}

\subsection{编译模板}
\label{sec:process}

模板默认使用GNUMake构建,GNUMake将调用latemk工具自动完成模板多轮编译:

\begin{lstlisting}[basicstyle=\small\ttfamily, caption={编译模板}, numbers=none]
make clean thesis.pdf
\end{lstlisting}

若需要生成包含“原创性声明扫描件”的学位论文文档,请将扫描件保存为statement.pdf,然后调用make生成submit.pdf。
在此过程中,stapler工具将被用于合并PDF文件。\emph{使用前请在Makefile中调整stapler合并的页码范围。}

\begin{lstlisting}[basicstyle=\small\ttfamily, caption={生成用于提交的学位论文}, numbers=none]
make clean submit.pdf
\end{lstlisting}


模板使用{\XeTeX}引擎提供的xelatex的命令处理,作用于“主控文档”thesis.tex。并且,可以省略扩展名。
在命令提示符下逐行敲入如下命令完成编译。

\begin{lstlisting}[basicstyle=\small\ttfamily, caption={手动执行编译过程}, numbers=none]
xelatex -no-pdf --interaction=nonstopmode thesis
bibtex thesis 
xelatex -no-pdf --interaction=nonstopmode thesis
xelatex --interaction=nonstopmode thesis 
\end{lstlisting}

\subsection{模板文件}
\label{sec:layout}

\begin{lstlisting}[basicstyle=\small\ttfamily,caption={模板文件布局},label=layout,float,numbers=none]
├── HOWTO.pdf
├── Makefile
├── README.md
├── bib
│   ├── chap1.bib
│   └── chap2.bib
├── bst
│   └── GBT7714-2005NLang.bst
├── figure
│   └── sjtubanner.png
│   └── chap2
│       ├── testeps.eps
│       ├── testjpg.jpg
│       ├── testpdf.pdf
│       └── testpng.png
├── sjtuthesis.cfg
├── sjtuthesis.cls
├── statement.pdf
├── tags
├── tex
│   ├── abstract.tex
│   ├── ack.tex
│   ├── app_eq.tex
│   ├── app_log.tex
│   ├── chapter01.tex
│   ├── chapter02.tex
│   ├── chapter03.tex
│   ├── conclusion.tex
│   ├── id.tex
│   ├── projects.tex
│   ├── pub.tex
│   └── symbol.tex
└── thesis.tex
\end{lstlisting}

几个主要文件的功能简述如下。

\subsubsection{格式控制文件}
\label{sec:format}

格式控制文件控制着论文的表现形式,包括以下几个文件:
sjtuthesis.cfg, sjtuthesis.cls和GBT7714-2005NLang.bst。
其中,“.cfg”和“.cls”控制论文主体格式,“.bst”控制参考文献条目的格式,

\subsubsection{主控文件thesis.tex}
\label{sec:thesistex}

主控文件thesis.tex的作用就是将你分散在多个文件中的内容“整合”成一篇完整的论文。
使用这个模板撰写学位论文时,你的学位论文内容和素材会被“拆散”到各个文件中:
譬如各章正文、各个附录、各章参考文献等等。
在thesis.tex中通过“include”命令将论文的各个部分包含进来,从而形成一篇结构完成的论文。
封面页中的论文标题、作者等中英文信息,也是在thesis.tex中填写。
部分可能会频繁修改的设置,譬如行间距、图片文件目录等,我也放在了thesis.tex中。
你也可以在thesis.tex中按照自己的需要引入一些的宏包
\footnote{我对宏包的态度是:只有当你需要在文档中使用那个宏包时,才需要在导言区中用usepackage引入该宏包。如若不然,通过usepackage引入一大堆不被用到的宏包,必然是一场灾难。由于一开始没有一致的设计目标,{\LaTeX}的各宏包几乎都是独立发展起来的,因重定义命令导致的宏包冲突屡见不鲜。}
。

大致而言,在thesis.tex中,大家只要留意把“章”一级的内容,以及各章参考文献内容包含进来就可以了。
需要注意,处理文档时所有的操作命令\cndash{}xelatex, bibtex等,都是作用在thesis.tex上,而\emph{不是}后面这些“分散”的文件,请参考\ref{sec:process}小节。

\subsubsection{论文主体文件夹body}
\label{sec:thesisbody}

这一部分是论文的主体,是以“章”为单位划分的。

正文前部分(frontmatter):中英文摘要(abstract.tex)。其他部分,诸如中英文封面、授权信息等,都是根据thesis.tex所填的信息“画”好了,
不单独弄成文件。

正文部分(mainmatter):自然就是各章内容chapter\emph{xxx}.tex了。

正文后的部分(backmatter):附录(app\emph{xx}.tex);致谢(thuanks.tex);攻读学位论文期间发表的学术论文目录(pub.tex);个人简历(resume.tex)。
参考文献列表是“生成”的,也不作为一个单独的文件。另外,学校的硕士研究生学位论文模板中,也没有要求加入个人简历,所以我没有在thesis.tex中引入resume.tex。

\subsubsection{图片文件夹figure}
\label{sec:fig}

fig文件夹放置了需要插入文档中的图片文件(PNG/JPG/PDF/EPS),建议按章再划分子目录。

\subsubsection{参考文献数据库文件夹bib}
\label{sec:bib}

reference文件夹放置的是各章“可能”会被引用的参考文献文件。
参考文献的元数据,例如作者、文献名称、年限、出版地等,会以一定的格式记录在纯文本文件.bib中。
最终的参考文献列表是BibTeX处理.bib后得到的,名为thesis.bbl。
将参考文献按章划分的一个好处是,可以在各章后生成独立的参考文献,不过,现在看来没有这个必要。
关于参考文献的管理,可以进一步参考第\ref{chap:example}章中的例子。


